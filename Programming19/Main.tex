\documentclass[english,submission]{programming}

\usepackage{mathtools}
\usepackage{caption}
\usepackage{wrapfig}
\usepackage{xspace}
\usepackage{xypic}
\usepackage{hyperref}
%\usepackage{amsmath}
\usepackage{url}
\usepackage{syntax}
\usepackage{textcomp}
\usepackage{color}
\usepackage{comment}
\newcommand{\dsl}{\textsc{Scans}\xspace}
\newcommand{\interp}{\textsc{Interpreter}\xspace}
\newcommand{\an}[3]{}
%\newcommand{\an}[3]{{\color{#2} {#1}:#3}}
\newcommand{\weixin}[1]{\an{weixin}{cyan}{#1}}
\newcommand{\bruno}[1]{\an{bruno}{blue}{#1}}

\DeclareOldFontCommand{\bf}{\normalfont\bfseries}{\mathbf}

\defcaptionname{english}{\sectionautorefname}{Section}
% for code that uses string literal in lhs2TeX
\DeclareOldFontCommand{\tt}{\normalfont\ttfamily}{\texttt}

\let\oldpar\paragraph
\renewcommand{\paragraph}[1]{\vspace{-7pt}\oldpar{#1}}
%\let\oldsec\section
%\renewcommand{\section}[1]{\vspace{-7pt}\oldsec{#1}}
\let\oldsubsec\subsection
\renewcommand{\subsection}[1]{\vspace{-7pt}\oldsubsec{#1}}

%% for comments
%%include lhs2TeX.fmt
%% ODER: format ==         = "\mathrel{==}"
%% ODER: format /=         = "\neq "
%
%
\makeatletter
\@ifundefined{lhs2tex.lhs2tex.sty.read}%
  {\@namedef{lhs2tex.lhs2tex.sty.read}{}%
   \newcommand\SkipToFmtEnd{}%
   \newcommand\EndFmtInput{}%
   \long\def\SkipToFmtEnd#1\EndFmtInput{}%
  }\SkipToFmtEnd

\newcommand\ReadOnlyOnce[1]{\@ifundefined{#1}{\@namedef{#1}{}}\SkipToFmtEnd}
\usepackage{amstext}
\usepackage{amssymb}
\usepackage{stmaryrd}
\DeclareFontFamily{OT1}{cmtex}{}
\DeclareFontShape{OT1}{cmtex}{m}{n}
  {<5><6><7><8>cmtex8
   <9>cmtex9
   <10><10.95><12><14.4><17.28><20.74><24.88>cmtex10}{}
\DeclareFontShape{OT1}{cmtex}{m}{it}
  {<-> ssub * cmtt/m/it}{}
\newcommand{\texfamily}{\fontfamily{cmtex}\selectfont}
\DeclareFontShape{OT1}{cmtt}{bx}{n}
  {<5><6><7><8>cmtt8
   <9>cmbtt9
   <10><10.95><12><14.4><17.28><20.74><24.88>cmbtt10}{}
\DeclareFontShape{OT1}{cmtex}{bx}{n}
  {<-> ssub * cmtt/bx/n}{}
\newcommand{\tex}[1]{\text{\texfamily#1}}	% NEU

\newcommand{\Sp}{\hskip.33334em\relax}


\newcommand{\Conid}[1]{\mathit{#1}}
\newcommand{\Varid}[1]{\mathit{#1}}
\newcommand{\anonymous}{\kern0.06em \vbox{\hrule\@width.5em}}
\newcommand{\plus}{\mathbin{+\!\!\!+}}
\newcommand{\bind}{\mathbin{>\!\!\!>\mkern-6.7mu=}}
\newcommand{\rbind}{\mathbin{=\mkern-6.7mu<\!\!\!<}}% suggested by Neil Mitchell
\newcommand{\sequ}{\mathbin{>\!\!\!>}}
\renewcommand{\leq}{\leqslant}
\renewcommand{\geq}{\geqslant}
\usepackage{polytable}

%mathindent has to be defined
\@ifundefined{mathindent}%
  {\newdimen\mathindent\mathindent\leftmargini}%
  {}%

\def\resethooks{%
  \global\let\SaveRestoreHook\empty
  \global\let\ColumnHook\empty}
\newcommand*{\savecolumns}[1][default]%
  {\g@addto@macro\SaveRestoreHook{\savecolumns[#1]}}
\newcommand*{\restorecolumns}[1][default]%
  {\g@addto@macro\SaveRestoreHook{\restorecolumns[#1]}}
\newcommand*{\aligncolumn}[2]%
  {\g@addto@macro\ColumnHook{\column{#1}{#2}}}

\resethooks

\newcommand{\onelinecommentchars}{\quad-{}- }
\newcommand{\commentbeginchars}{\enskip\{-}
\newcommand{\commentendchars}{-\}\enskip}

\newcommand{\visiblecomments}{%
  \let\onelinecomment=\onelinecommentchars
  \let\commentbegin=\commentbeginchars
  \let\commentend=\commentendchars}

\newcommand{\invisiblecomments}{%
  \let\onelinecomment=\empty
  \let\commentbegin=\empty
  \let\commentend=\empty}

\visiblecomments

\newlength{\blanklineskip}
\setlength{\blanklineskip}{0.66084ex}

\newcommand{\hsindent}[1]{\quad}% default is fixed indentation
\let\hspre\empty
\let\hspost\empty
\newcommand{\NB}{\textbf{NB}}
\newcommand{\Todo}[1]{$\langle$\textbf{To do:}~#1$\rangle$}

\EndFmtInput
\makeatother
%
%
%
%
%
%
% This package provides two environments suitable to take the place
% of hscode, called "plainhscode" and "arrayhscode". 
%
% The plain environment surrounds each code block by vertical space,
% and it uses \abovedisplayskip and \belowdisplayskip to get spacing
% similar to formulas. Note that if these dimensions are changed,
% the spacing around displayed math formulas changes as well.
% All code is indented using \leftskip.
%
% Changed 19.08.2004 to reflect changes in colorcode. Should work with
% CodeGroup.sty.
%
\ReadOnlyOnce{polycode.fmt}%
\makeatletter

\newcommand{\hsnewpar}[1]%
  {{\parskip=0pt\parindent=0pt\par\vskip #1\noindent}}

% can be used, for instance, to redefine the code size, by setting the
% command to \small or something alike
\newcommand{\hscodestyle}{}

% The command \sethscode can be used to switch the code formatting
% behaviour by mapping the hscode environment in the subst directive
% to a new LaTeX environment.

\newcommand{\sethscode}[1]%
  {\expandafter\let\expandafter\hscode\csname #1\endcsname
   \expandafter\let\expandafter\endhscode\csname end#1\endcsname}

% "compatibility" mode restores the non-polycode.fmt layout.

\newenvironment{compathscode}%
  {\par\noindent
   \advance\leftskip\mathindent
   \hscodestyle
   \let\\=\@normalcr
   \(\pboxed}%
  {\endpboxed\)%
   \par\noindent
   \ignorespacesafterend}

\newcommand{\compaths}{\sethscode{compathscode}}

% "plain" mode is the proposed default.

\newenvironment{plainhscode}%
  {\hsnewpar\abovedisplayskip
   \advance\leftskip\mathindent
   \hscodestyle
   \let\\=\@normalcr
   \(\pboxed}%
  {\endpboxed\)%
   \hsnewpar\belowdisplayskip
   \ignorespacesafterend}

% Here, we make plainhscode the default environment.

\newcommand{\plainhs}{\sethscode{plainhscode}}
\plainhs

% The arrayhscode is like plain, but makes use of polytable's
% parray environment which disallows page breaks in code blocks.

\newenvironment{arrayhscode}%
  {\hsnewpar\abovedisplayskip
   \advance\leftskip\mathindent
   \hscodestyle
   \let\\=\@normalcr
   \(\parray}%
  {\endparray\)%
   \hsnewpar\belowdisplayskip
   \ignorespacesafterend}

\newcommand{\arrayhs}{\sethscode{arrayhscode}}

% The mathhscode environment also makes use of polytable's parray 
% environment. It is supposed to be used only inside math mode 
% (I used it to typeset the type rules in my thesis).

\newenvironment{mathhscode}%
  {\parray}{\endparray}

\newcommand{\mathhs}{\sethscode{mathhscode}}

% texths is similar to mathhs, but works in text mode.

\newenvironment{texthscode}%
  {\(\parray}{\endparray\)}

\newcommand{\texths}{\sethscode{texthscode}}

% The framed environment places code in a framed box.

\def\codeframewidth{\arrayrulewidth}
\RequirePackage{calc}

\newenvironment{framedhscode}%
  {\parskip=\abovedisplayskip\par\noindent
   \hscodestyle
   \arrayrulewidth=\codeframewidth
   \tabular{@{}|p{\linewidth-2\arraycolsep-2\arrayrulewidth-2pt}|@{}}%
   \hline\framedhslinecorrect\\{-1.5ex}%
   \let\endoflinesave=\\
   \let\\=\@normalcr
   \(\pboxed}%
  {\endpboxed\)%
   \framedhslinecorrect\endoflinesave{.5ex}\hline
   \endtabular
   \parskip=\belowdisplayskip\par\noindent
   \ignorespacesafterend}

\newcommand{\framedhslinecorrect}[2]%
  {#1[#2]}

\newcommand{\framedhs}{\sethscode{framedhscode}}

% The inlinehscode environment is an experimental environment
% that can be used to typeset displayed code inline.

\newenvironment{inlinehscode}%
  {\(\def\column##1##2{}%
   \let\>\undefined\let\<\undefined\let\\\undefined
   \newcommand\>[1][]{}\newcommand\<[1][]{}\newcommand\\[1][]{}%
   \def\fromto##1##2##3{##3}%
   \def\nextline{}}{\) }%

\newcommand{\inlinehs}{\sethscode{inlinehscode}}

% The joincode environment is a separate environment that
% can be used to surround and thereby connect multiple code
% blocks.

\newenvironment{joincode}%
  {\let\orighscode=\hscode
   \let\origendhscode=\endhscode
   \def\endhscode{\def\hscode{\endgroup\def\@currenvir{hscode}\\}\begingroup}
   %\let\SaveRestoreHook=\empty
   %\let\ColumnHook=\empty
   %\let\resethooks=\empty
   \orighscode\def\hscode{\endgroup\def\@currenvir{hscode}}}%
  {\origendhscode
   \global\let\hscode=\orighscode
   \global\let\endhscode=\origendhscode}%

\makeatother
\EndFmtInput
%
\begin{CCSXML}
<ccs2012>
<concept>
<concept_id>10011007.10011006.10011008.10011024</concept_id>
<concept_desc>Software and its engineering~Language features</concept_desc>
<concept_significance>500</concept_significance>
</concept>
<concept>
<concept_id>10011007.10011006.10011050.10011017</concept_id>
<concept_desc>Software and its engineering~Domain specific languages</concept_desc>
<concept_significance>500</concept_significance>
</concept>
</ccs2012>
\end{CCSXML}

\ccsdesc[500]{Software and its engineering~Language features}
\ccsdesc[500]{Software and its engineering~Domain specific languages}
% Document starts
\begin{document}

% Title portion
\title{Shallow EDSLs and Object-Oriented Programming}
\subtitle{Beyond Simple Compositionality}

\author[a]{Weixin Zhang}
\author[a]{Bruno C. d. S. Oliveira}
\affiliation[a]{The University of Hong Kong, Hong Kong, China}

\paperdetails{
  %% perspective options are: art, sciencetheoretical, scienceempirical, engineering.
  %% Choose exactly the one that best describes this work. (see 2.1)
  perspective=art,
  %% State one or more areas, separated by a comma. (see 2.2)
  %% Please see list of areas in http://programming-journal.org/cfp/
  %% The list is open-ended, so use other areas if yours is/are not listed.
  area={Domain-Specific Languages, Modularity and Separation of Concerns},
  %% You may choose the license for your paper (see 3.)
  %% License options include: cc-by (default), cc-by-nc
  % license=cc-by,
}

\maketitle

\begin{abstract}

{\bf Context.} 
Embedded Domain-Specific Languages (EDSLs) are a common and widely
used approach to
DSLs in various languages, including Haskell and Scala. There are two
main implementation techniques for EDSLs: \emph{shallow embeddings}
and \emph{deep embeddings}. 

{\bf Inquiry.} Shallow embeddings are quite simple, 
but they have been criticized in the past for being quite limited in
terms of modularity and reuse. In particular, it is often argued that
supporting multiple DSL interpretations in shallow embeddings is difficult.


{\bf Approach.}
This paper argues that shallow EDSLs and Object-Oriented
Programming (OOP) are closely related. Gibbons and Wu already
discussed the relationship between shallow EDSLs and procedural
abstraction, while Cook discussed the connection between procedural
abstraction and OOP. We make the transitive step in this paper by
connecting shallow EDSLs directly to OOP via procedural abstraction. 
The knowledge about this relationship enables us to
improve on implementation techniques for EDSLs.

{\bf Knowledge.}
This paper argues that common OOP mechanisms
(including \emph{inheritance}, \emph{subtyping}, and
\emph{type-refinement})
increase the modularity and reuse of shallow
EDSLs when compared to classical procedural abstraction by enabling a
simple way to express \emph{multiple, possibly dependent,
  interpretations}. 

{\bf Grounding.}
We make our arguments by using Gibbons and Wu's examples, where procedural
abstraction is used in Haskell to model a simple shallow EDSL. We recode that
EDSL in Scala and with an improved OO-inspired Haskell encoding.
We further illustrate our approach with
a case study on refactoring a deep external SQL DSL implementation to make it
more modular, shallow, and embedded.

{\bf Importance.} This work is important for two reasons.
Firstly, from an intellectual point of view, this work
establishes the connection between shallow embeddings and OOP, which enables a
better understanding of both concepts. Secondly, this work
illustrates programming techniques that can be used to improve 
the modularity and reuse of shallow EDSLs.  
\end{abstract}

\section{Introduction}

Since Hudak's seminal paper on \weixin{EDSLs}~\cite{hudak1998modular}, existing
languages (such as Haskell) have been used to directly encode
DSLs. Two common approaches to EDSLs are the so-called \emph{shallow}
and \emph{deep} embeddings. The origin of that terminology can be
attributed to Boulton et al.'s work~\cite{Boulton92dsl}. The difference between these
two styles of embeddings is commonly described as follows:

\begin{quote}
\emph{With a deep embedding, terms in the DSL are implemented simply to
construct an abstract syntax tree (AST), which is subsequently
transformed for optimization and traversed for evaluation. With a
shallow embedding, terms in the DSL are implemented directly by
their semantics, bypassing the intermediate AST and its traversal.}\cite{gibbons2014folding}
\end{quote}

%\begin{comment}
%This definition is widely accepted and similar definitions appear in
%many other works~\cite{}. 
%We argue that this definition is vague,
%and often leads to some contradicting claims. 

Although the above definition is quite reasonable and widely accepted,
it leaves some space to (mis)interpretation. For example it is unclear 
how to classify an EDSL implemented using the {\sc Composite} or {\sc Interpreter} 
patterns in OOP. Would this OO approach be
classified as a shallow or deep embedding? We believe arguments can be
made both ways. The {\sc Composite} or {\sc Interpreter}
patterns are normally accepted to provide a way to encode ASTs. Thus, 
one possible interpretation is that \emph{according to definition of deep embedding
  above, the OO approach classifies as a deep
  embedding}. 

\begin{comment}
For example, in their work
on EDSLs~\cite{}, Gibbons and Wu claim that deep embeddings (which
encode ASTs using algebraic datatypes in Haskell) allow adding new DSL
interpretations easily, but they make adding new language constructs
difficult. In contrast Gibbons and Wu claim that shallow embeddings
have dual modularity properties: new cases are easy to add, but new
interpretations are hard.  However what if, instead of using Haskell
and algebraic datatypes, one uses an OO language to encode an AST, for
example with the {\sc Composite} pattern.  Would this OO approach be
classified as a shallow or deep embedding? We believe arguments can be
made both ways. Since the {\sc Composite}
pattern is normally accepted to be a way to encode ASTs, it would be
reasonable to say that \emph{according to definition of deep embedding
  above, the OO approach classifies as a deep
  embedding}. Unfortunatelly this interpretation could be problematic.
As the Expression Problem~\cite{} tell us,
in the OO approach adding new language constructs is easy, but adding
interpretations is hard. Thus this would contradict Gibbons and Wu's
claims, since we have an AST representation (i.e. a deep embedding)
with the modularity properties of shallow embeddings.

We believe that the core of problem is that ASTs can be represented in
multiple ways. In particular, it is well know that functions alone are
enough to encode datastructures such as ASTs (via Church
encodings~\cite{}).  Distinguishing deep and shallow embeddings based
solely on whether a ``real'' datastructure is being used or not is
misleading.  Moreover, it gives the impression that shallow embeddings
are significantly less expressive than deep embeddings, because they
do not have access to the datastructure.
Gibbons and Wu themselves feel uneasy with the definition of shallow 
embeddings when they say:
``\emph{So it turns out that the syntax of the DSL is not really as ephemeral
in a shallow embedding as Boulton's choice of terms suggests.}''
\end{comment}

In this paper, to avoid ambiguity, we prefer a more precise meaning
for shallow embeddings as EDSLs implemented using \emph{procedural abstraction}~\cite{reynolds94proceduralabstraction}. Such
interpretation arises from the domain of shallow DSLs being
typically a function, and procedural abstraction being a way to encode
data abstractions using functions. As Cook~\cite{cook09abstraction} argued,
procedural abstraction is also the essence of Object-Oriented
Programming. Thus, according to our definition, the implementation of a shallow
EDSL in OOP languages should simply correspond to a standard
object-oriented program.

\begin{comment}
If we accept Cook's view on OOP, 
a natural way to distinguish implementations of EDSLs is 
in terms of the data abstraction used to model the language
constructs instead. As Reynold's~\cite{reynolds94proceduralabstraction} suggested there are two
types of data abstraction: procedural abstraction and \emph{user-defined
  types}. It is clear that shallow embeddings use \emph{procedural
  abstraction}: the DSLs are modelled by interpretation
functions. Thus, the other implementation option for EDSLs is to
use \emph{user-defined types}. In Reynolds terminology user-defined
types mean disjoint union types, which are nowadays commonly available
in modern languages as \emph{algebraic datatypes}. Disjoint union
types can also be emulated in OOP using the {\sc Visitor} pattern. 

A distinction based on data abstraction is more precise and provides a
remedy for possible misinterpretation. An EDSL implemented with
algebraic datatypes falls into the category of user-defined types
(deep embedding), while a {\sc Composite}-based OO implementation falls
under procedural abstraction (shallow embedding). 
\end{comment}

%At least some authors~\cite{} seem to implicitly
%agree with the latter interpretation.

The main goal of this paper is to argue that OOP languages have
advantages for the implementation of shallow embeddings. 
%This should
%not came as a surprise, as OOP languages have evolved over more than 50
%years to improve the use of procedural abstraction. 
In order to understand how OOP can help with shallow embeddings, lets
first review some of the limitations commonly
found in the literature:

\begin{enumerate}

\item {\bf Single Interpretation} An often stated limitation of
  shallow embeddings is that they only support a single
  interpretation.

\item {\bf No Transformations} Another commonly stated limitation 
of shallow embeddings is that they do not support transformations,
preventing optimizations and other useful transformations.

\end{enumerate}

\noindent Both limitations are often used as motivators to switch to deep embeddings.

We show that OOP abstractions, including \emph{(multiple)
  inheritance}, \emph{subtyping} and \emph{type-refinement}, are
helpful to address those problems. For the first problem, we can
employ a recently proposed design pattern~\cite{eptrivially16}, which provides a simple
solution to the \emph{Expression Problem}~\cite{expPb} in OOP languages. Thus
using just standard OOP mechanims enables \emph{multiple modular
  interpretations} to co-exist and be combined in shallow embeddings.
For the second problem, we show that transformations can be encoded 
with recursive objects. Interestingly enough, for transformations it
is possible to port back the OO solution to
Haskell.

We make our arguments by taking a recent paper by Gibbons and Wu~\cite{gibbons2014folding},
where procedural abstraction is used in Haskell to model a simple
EDSL, and we recode that EDSL in Scala. Although from the
\emph{syntactical} point of view there are minor inconveniences in
the Scala version, from the \emph{semantic} and \emph{modularity} point of view the
Scala version has clear advantages.

\begin{comment}
In summary, our contributions are:

\begin{itemize}

\item {}

\item {\bf Multiple Modular Interpretations for Shallow Embeddings:} 
  We show that with standard OOP mechanisms it is easy to support multiple modular
  interpretations for shallow embeddings.

\item {\bf Transformations for Shallow Embeddings:} We show that
  transformations are encodable with recursive objects. Moreover, this technique
  can be ported back into functional programming as well.

\end{itemize}
\end{comment}


\section{Shallow Object-Oriented Programming}

Argue that shallow embeddings and straightforward OO 
programs are essentially the same thing. 

Start from a simple shallow DSL in Haskell, 
and iterate throught it until you reach a form 
that looks like an OO program.

Show how todo transformations in Shallow embeddings
using the insight of how to do transformations in OO
programs.

Show the correponding Java programs and the Java program 
with transformation that we can port back to Haskell.

\begin{verbatim}
{-
type Exp = Int

lit x = x

add e1 e2 = e1 + e2
-}

{-
newtype Exp = E Int

lit x = E x

add (E e1) (E e2) = E (e1 + e2)
-}

data Exp = E {
  eval :: Int,
  transform :: Exp
}

lit x = E {eval = x, transform = lit x}

add e1 e2 = E {
  eval = eval e1 + eval e2,
  transform = add (lit 0) e2
}

p = eval $ transform (add (lit 5) (lit 3))
\end{verbatim}


\section{Interpretations in Shallow Embeddings}

A often stated limitation of shallow embeddings is that they allow only a single
interpretation. Gibbons and Wu~\cite{gibbons2014folding}  work around this problem by using tuples. However, their encoding needs to modify
the original code, and thus is non-modular. This section illustrates how various type of
interpretations can be \emph{modularly} defined in OOP.
\begin{comment}
Although a modular solution based on \cite{swierstra2008data}
is also presented, it complicates the encoding dramatically and may prevent pratical use.
OO approach, on the contrary, provides modular yet simple solution of defining
multiple interpretations. 
\end{comment}


\subsection{Multiple Interpretations}\label{subsec:multiple}
%Defining additional interpretations is not trivial for shallow embedding,
%especially for functional languages.
\paragraph{Multiple Interpretations in Haskell}
Suppose that we want to have an additional function that checks whether a circuit is
constructed correctly. Gibbons and Wu's solution is:
\lstinputlisting[linerange=2-9]{./code/NewSemantics.hs}%APPLY:SEMANTICS_HS
\noindent This solution is not modular because it relies 
on defining the two interpretations (\lstinline{width} and
\lstinline{wellSized}) simultaneously, using a tuple. It is not
possible reuse the independently defined \lstinline{width} function in
Section~\ref{subsec:shallow}.
Whenever a new interpretation is needed (e.g. \lstinline{wellSized}), the
original code has to be revised:
the arity of the tuple must be incremented and the new interpretation has to be
appended to each case.
%we add the definition of \lstinline{wellSized} by modifying the original code.


\paragraph{Multiple Interpretations in Scala}
In contrast, Scala allows new interpretations to be introduced in a 
modular way, as shown in Figure~\ref{code:operation}.
\begin{figure}
\lstinputlisting[linerange=2-15]{../src/wellsized/Circuit.scala}%APPLY:MULTIPLE_SCALA
\caption{Adding new interpretations}
\label{code:operation}
\end{figure}
%Instead of modifying the original code, we define \lstinline{wellSized} modularly.
The encoding relies on three OOP abstraction mechanisms:
\emph{inheritance}, \emph{subtyping} and \emph{type-refinement}.
Specifically, the new \lstinline{Circuit} is a subtype of
\lstinline{base.Circuit} and declares a new method \lstinline{wellSized}.
The hierarchy implements the new \lstinline{Circuit} by inheriting the corresponding class
from \lstinline{base} and
implementing \lstinline{wellSized}.
Also, fields of \lstinline{Beside} are refined with the new \lstinline{Circuit} type
to avoid type mismatches in methods~\cite{eptrivially16}.

% Multiple inheritance
\begin{comment}
We can even define \lstinline{wellSized} independently:
\begin{lstlisting}
trait Circuit { def wellSized: Boolean }
trait Id extends Circuit { ... }
...
\end{lstlisting}
And merge the two hierarchies through \emph{multiple inheritance} for providing
multiple interpretations:
\begin{lstlisting}
trait Circuit
    extends width.Circuit with wellSized.Circuit
trait Id extends Circuit
    with width.Id with wellSized.Circuit
...
\end{lstlisting}
\end{comment}

\subsection{Dependent Interpretations}
 \emph{Dependent interpretations} are a generalization of multiple
interpretations, where one or more interpretations can use other
interpretations in their definition. 
In Haskell dependent interpretations are once again defined with
tuples in a non-modular way. We omit the Haskell code here, since 
it is quite similar to the example in Section~\ref{subsec:multiple}.

%Such interpretations are non-compositional.
%In Haskell a dependent interpretation must be defined together with what it
%dependents on and makes no exceptions on modular approaches like~\cite{}.
%This prevents a new interpretation that depends on existing
%interpretations from being defined modularly.
%Fortunately, OO approach does not have such restriction.

To illustrate dependent interpretations, we first add two new language 
constructs \emph{above} and \emph{stretch}:
\setlength{\grammarindent}{5em} % increase separation between LHS/RHS
%\lstinputlisting[linerange=29-30]{./code/shallowCircuit.hs}%APPLY:SYNTAX_TYPES
\begin{grammar}
<circuit> ::= \ldots
\alt `above' <circuit> <circuit>
\alt `stretch' <positive-numbers> <circuit>
\end{grammar}
$above\ c_1\ c_2$ combines two circuits of the same width vertically;
\emph{stretch ns c} inserts additional wires into the circuit \emph{c} by
summing up \emph{ns}.
Figure~\ref{code:variant} shows how to add them to \dsl, simply by
\emph{modularly} defining new classes that implement \lstinline{Circuit}.
\begin{figure}
\lstinputlisting[linerange=2-11]{../src/width/Variants.scala}%APPLY:VARIANT_SCALA
\caption{Adding new language constructs}
\label{code:variant}
\end{figure}

\paragraph{Dependent Interpretations in Scala}
The definitions of \lstinline{wellSized} for $above$ and $stretch$
require dependent interpretations.
Circuits written with $above$ and $stretch$ have to obey to certain
size constrains, which are captured by the \lstinline{wellSized} method:
\lstinputlisting[linerange=2-11]{../src/wellsized/Variants.scala}%APPLY:DEPENDENT_SCALA
The definitions of \lstinline{wellSized} for extended cases depend on \lstinline{width}.
Note that \lstinline{width} and \lstinline{wellSized} are defined separately.
Essentially, it is sufficient to define \lstinline{wellSized} while
knowing only the signature of \lstinline{width} in \lstinline{Circuit}.

\subsection{Discussion} 
Gibbons and Wu claim that in shallow
embeddings new language constructs are easy to add, but new
interpretations are hard. As our OOP approach shows, in OOP both
language constructs and new interpretations are easy to add in shallow
embeddings. In other words, the circuit DSL presented so far does not
suffer from the Expression Problem. The key point is that procedural
abstraction combined with OOP features (subtyping, inheritance and
type-refinement) adds expressiveness over traditional procedural
abstraction. Gibbons and Wu do discuss a number of advanced techniques that 
can solve some of the modularity problems. For example, using type
classes, \emph{finally
  tagless}~\cite{carette2009finally} can deal with the example in
Section~\ref{subsec:multiple}. However
tuples are still needed 
to deal with dependent interpretations. In contrast the approach
proposed here is just straightforward OOP, and dependent
interpretations are not a problem.
\begin{comment}
and \emph{data types a la
  carte}~\cite{swierstra2008data} (DTC).
Finally tagless approach uses a type class to abstract over all interpretations
of the language. Concrete interpretations are given through creating a data type and
making it an instance of that type. However, it forces dependent interpretations to be defined along with what they depend on.
DTC represents language constructs separately and composes them together using
extensible sums. However, not like OO languages which come with subtyping, one
has to manually implement the subtyping machinery for variants.
\end{comment}
Gibbons and Wu also show some different variants of interpretations,
such as context-sensitive interpretations. 
Context-sensitive transformations are unproblematic as well. 
For space reasons, full details are only available online.

\begin{comment}
Unlike \lstinline{width} and \lstinline{wellSized} which can be defined with
only the given circuit, context-dependent interpretations may need some context.
These contexts can be captured by arguments of the method. For example, a
function that collects all the connections between wires inside a circuit would have
the following signature:
\begin{lstlisting}
type Layout = List[List[Tuple2[Int,Int]]]
def tlayout(f: Int => Int): Layout
\end{lstlisting}
where the context \lstinline{f} may vary in recursive
calls. Context-sensitive transformations do not pose any particular
challenge. For space reasons, we omit the implementation details here. Full details
are available online.
\end{comment}

\input{sections/TypeClass}

\input{sections/Factory}

\input{sections/Casestudy}

\section{Discussion and Conclusion}
This paper shows how OO programming improves the modularity of shallow EDSLs.
With the procedural abstraction mechanisms provided by OO languages, various types of
interpretations can be defined modularly. We also show that defining
transformations in shallow embeddings is possible.


Existing work showed that shallow embeddings yield flexible and concise EDSLs, while deep embedding makes
it easy to define optimizations.
There is a lot of work~\cite{svenningsson2012combining,
  Jovanovic:2014:YCD:2658761.2658771, scherr2014implicit} trying to blend these two
approaches to enjoy benefits from both.
They typically encode the surface language in shallow embeddings and
then generate or translate to a deep embedded version for allowing optimizations.
%Hofer and Ostermann~\cite{hofer2010modular} propose to provide both embedding through implementing internal and external visitor at the same time so that clients can choose for a particular interpretation;
Our OO approach retains the simplicity of shallow embeddings while
allowing optimizations. It would be interesting
to conduct larger case studies to assess whether the techniques
presented here are enough to avoid deep embeddings for various DSLs
in the literature.

One limitation of our approach is that
transformations require code duplication in extensions,
as we must give an implementation when refining their return type.
Though transformations can still be defined modularly with advanced type
system features of Scala~\cite{zenger05independentlyextensible},
the encoding would become significantly more complicated.
How to define these transformations modularly, without code
duplication, and without sophisticated types
is a possible line of future work.  Also, it is tedious to
express the inheritance relationships in extensions, especially when
multiple inheritance is used. Another line of future work is to use
some meta-programming mechanisms to eliminate such boilerplate.


% Bibliography
%\renewcommand{\section}[1]{\oldsec{#1}}
\acks We thank Willam R. Cook, Jeremy Gibbons, Ralf Hinze, Martin Erwig, and the
anonymous reviewers of GPCE, ICFP, JFP, and Programming for their valuable
comments that significantly improved this work. This work is based on the first
author's master thesis~\cite{zhang2017extensible} and has been funded by Hong
Kong Research Grant Council projects number 17210617 and 17258816.

\bibliography{Main}

% History dates
%\received{February 2007}{March 2009}{June 2009}

\end{document}
