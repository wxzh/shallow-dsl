\documentclass[english,submission]{programming}

\usepackage{mathtools}
\usepackage{caption}
\usepackage{wrapfig}
\usepackage{xspace}
\usepackage{xypic}
\usepackage{hyperref}
%\usepackage{amsmath}
\usepackage{url}
\usepackage{syntax}
\usepackage{textcomp}
\usepackage{color}
\usepackage{comment}
\newcommand{\dsl}{\textsc{Scans}\xspace}
\newcommand{\interp}{\textsc{Interpreter}\xspace}
\newcommand{\an}[3]{{\color{#2} {\sc #1}:#3}}
%\newcommand{\an}[3]{}
\newcommand{\weixin}[1]{\an{weixin}{cyan}{#1}}
\newcommand{\bruno}[1]{\an{bruno}{blue}{#1}}

\renewcommand*{\sectionautorefname}{Section}
% for code that uses string literal in lhs2TeX
\DeclareOldFontCommand{\tt}{\normalfont\ttfamily}{\texttt}

\let\oldpar\paragraph
\renewcommand{\paragraph}[1]{\vspace{-7pt}\oldpar{#1}}
%\let\oldsec\section
%\renewcommand{\section}[1]{\vspace{-7pt}\oldsec{#1}}
\let\oldsubsec\subsection
\renewcommand{\subsection}[1]{\vspace{-7pt}\oldsubsec{#1}}

%% for comments
\input{Preamble}

% Document starts
\begin{document}

% Title portion
\title{Shallow EDSLs and Object-Oriented Programming}
\subtitle{Beyond Simple Compositionality}

\author[a]{Weixin Zhang}
\author[a]{Bruno C. d. S. Oliveira}
\affiliation[a]{The University of Hong Kong, Hong Kong, China}
%\author[W. Zhang and B. Oliveira]
%        {WEIXIN ZHANG and BRUNO C. D. S. OLIVEIRA\\
%         The University of Hong Kong, Hong Kong}

\paperdetails{
  %% perspective options are: art, sciencetheoretical, scienceempirical, engineering.
  %% Choose exactly the one that best describes this work. (see 2.1)
  perspective=art,
  %% State one or more areas, separated by a comma. (see 2.2)
  %% Please see list of areas in http://programming-journal.org/cfp/
  %% The list is open-ended, so use other areas if yours is/are not listed.
  area={Domain-Specific Languages, Modularity and separation of concerns},
  %% You may choose the license for your paper (see 3.)
  %% License options include: cc-by (default), cc-by-nc
  % license=cc-by,
}

\maketitle

\begin{abstract}
Shallow embedded domain-specific languages (EDSLs) use
\emph{procedural abstraction} to directly encode a DSL into an
existing host language. Procedural abstraction has 
been argued to be the essence of object-oriented programming (OOP).
This pearl argues that OOP abstractions
(including \emph{inheritance}, \emph{subtyping}, and
\emph{type-refinement})
increase the modularity and reuse of shallow
EDSLs when compared to classical procedural abstraction by enabling a
simple way to express \emph{multiple, possibly dependent, interpretations}. We make this
argument by taking a recent paper by Gibbons and Wu, where procedural
abstraction is used in Haskell to model a simple shallow EDSL, and we recode
that EDSL in Scala. 
We further illustrate our functional OOP approach with
a case study on refactoring a deep external SQL DSL implementation to make it
more modular, shallow, and embedded.
\end{abstract}

\section{Introduction}

Since Hudak's seminal paper on Embedded DSLs (EDSLs)~\cite{}, existing
languages (such as Haskell) have been used to directly encode
DSLs. Two common approaches to EDSLs are the so-called \emph{shallow}
and \emph{deep} embeddings. The origin of that terminology can be
attributed to Boulton's work~\cite{}. The difference between these
two styles of embeddings is commonly described as follows:

\begin{quote}
\emph{With a deep embedding, terms in the DSL are implemented simply to
construct an abstract syntax tree (AST), which is subsequently
transformed for optimization and traversed for evaluation. With a
shallow embedding, terms in the DSL are implemented directly by
their semantics, bypassing the intermediate AST and its traversal.}\cite{gibbons15folding}
\end{quote}

%\begin{comment}
%This definition is widely accepted and similar definitions appear in
%many other works~\cite{}. 
%We argue that this definition is vague,
%and often leads to some contradicting claims. 

Although the above definition is quite reasonable and widely accepted,
it leaves some space to (mis)interpretation. For example it is unclear 
how to classify an EDSL using the {\sc Composite} or {\sc Interpreter} 
patterns in OOP. Would this OO approach be
classified as a shallow or deep embedding? We believe arguments can be
made both ways. Since the {\sc Composite}
pattern is normally accepted to be a way to encode ASTs, it would be
reasonable to say that \emph{according to definition of deep embedding
  above, the OO approach classifies as a deep
  embedding}. However, as we shall argure in the remainder of the
paper, another possible interpretation is that the OO approach is
really a shallow embedding. At least some authors~\cite{} seem to agree with 
the latter interpretation, but we believe that the current definition
leaves some space open to interpretation.

\begin{comment}
For example, in their work
on EDSLs~\cite{}, Gibbons and Wu claim that deep embeddings (which
encode ASTs using algebraic datatypes in Haskell) allow adding new DSL
interpretations easily, but they make adding new language constructs
difficult. In contrast Gibbons and Wu claim that shallow embeddings
have dual modularity properties: new cases are easy to add, but new
interpretations are hard.  However what if, instead of using Haskell
and algebraic datatypes, one uses an OO language to encode an AST, for
example with the {\sc Composite} pattern.  Would this OO approach be
classified as a shallow or deep embedding? We believe arguments can be
made both ways. Since the {\sc Composite}
pattern is normally accepted to be a way to encode ASTs, it would be
reasonable to say that \emph{according to definition of deep embedding
  above, the OO approach classifies as a deep
  embedding}. Unfortunatelly this interpretation could be problematic.
As the Expression Problem~\cite{} tell us,
in the OO approach adding new language constructs is easy, but adding
interpretations is hard. Thus this would contradict Gibbons and Wu's
claims, since we have an AST representation (i.e. a deep embedding)
with the modularity properties of shallow embeddings.

We believe that the core of problem is that ASTs can be represented in
multiple ways. In particular, it is well know that functions alone are
enough to encode datastructures such as ASTs (via Church
encodings~\cite{}).  Distinguishing deep and shallow embeddings based
solely on whether a ``real'' datastructure is being used or not is
misleading.  Moreover, it gives the impression that shallow embeddings
are significantly less expressive than deep embeddings, because they
do not have access to the datastructure.
Gibbons and Wu themselves feel uneasy with the definition of shallow 
embeddings when they say:
``\emph{So it turns out that the syntax of the DSL is not really as ephemeral
in a shallow embedding as Boulton's choice of terms suggests.}''
\end{comment}

To avoid confusion and make the terminology more precise, we first 
propose distinguishing EDSLs in terms of the data
abstraction used to model the language constructs instead.  We follow
Reynold's classification~\cite{} of data abstractions:
\emph{procedural abstraction} and \emph{user-defined types}. It is
clear that shallow embeddings use \emph{procedural
  abstraction}~\cite{}: the DSLs are modelled by interpretation
functions. Therefore, the other implementation option for EDSLs is to use
\emph{user-defined types}. In Reynolds terminology user-defined types
mean disjoint union types, which are nowadays commonly available in
modern languages as \emph{algebraic datatypes}. Disjoint union types 
can also be emulated in OOP using the {\sc Visitor} pattern. A
distinction based on data abstraction provides a remedy for possible
misinterpretation. An EDSL implemented with algebraic
datatypes falls into the category of user-defined types (deep embedding), 
while a Composite-based OO implementation falls under procedural
abstraction (shallow embedding). For the rest of the paper we identify shallow EDSLs 
with EDSLs implemented using procedural abstraction.

This paper shows that OOP languages have
advantages for the implementation of shallow embeddings.
As Cook~\cite{} argued procedural abstraction is the essence of
Object-Oriented Programming. If we accept Cook's view, the
implementation of a shallow DSL in OOP languages should simply
correspond to a standard object-oriented program. Given
that Object-Oriented Languages have evolved over more than 50 years 
to improve the use of procedural abstraction, they ought to have some 
advantages to encode shallow EDSLs. 

We show that OOP abstractions, including \emph{inheritance}
and \emph{subtyping}, increase the modularity and reuse of shallow
DSLs when compared to classical procedural abstraction. We make this
argument by taking a recent paper by Gibbons and Wu, where procedural
abstraction is used in Haskell to model a simple EDSL, and we recode
that EDSL in Java. Although from the \emph{syntactical} point of view
there are obvious disadvantages in the Java version, from the semantic
and modularity point of view the Java version has clear advantages.


This paper has two goals:

\begin{itemize}

\item To argue that 

\end{itemize} 

\section{Shallow Object-Oriented Programming}\label{sec:oo}

\begin{comment}
Weixin writes this part.

Argue that shallow embeddings and straightforward OO 
programs are essentially the same thing. 

Start from a simple shallow DSL in Haskell, 
and iterate throught it until you reach a form 
that looks like an OO program.

Show how todo transformations in Shallow embeddings
using the insight of how to do transformations in OO
programs.

Show the correponding Java programs and the Java program 
with transformation that we can port back to Haskell.
\end{comment}

This section shows that an OO approach and shallow embeddings using
procedural abstraction are closely related.  We use a subset of the
DSL presented in Gibbons and Wu's paper~\cite{gibbons2014folding} as
the running example.  We first give the original shallow embedded
implementation in Haskell and rewrite it towards an ``OO style''.
Then translating the program into an OO language becomes straightforward.

\subsection{A DSL for Parallel Prefix Circuits}
Consider a DSL named \dsl that models parallel circuits.
Its BNF grammar is given below:
% BNF grammar should not contain left recursion and left factor, rewrite it
% using descent recursion
% \setlength{\grammarparsep}{20pt plus 1pt minus 1pt} % increase separation between rules
\setlength{\grammarindent}{5em} % increase separation between LHS/RHS

\begin{grammar}
<circuit> ::= `fan' <positive-number>
\alt `id' <positive-number>
\alt `beside' <circuit> <circuit>
\end{grammar}

\noindent The DSL has three constructs: two primitives
\emph{id} and \emph{fan} and one combinator \emph{beside}.
Their meanings are: \emph{id n} contains \emph{n} isolated vertical wires;
\emph{fan n} has \emph{n} vertical wires with its first wire connected to
all the remaining wires from top to bottom; $beside\ c_1\ c_2$ joins two circuits
$c_1$ and $c_2$ horizontally. Simple circuits can be described with these three constructs.
%For example, Figure~\ref{} visualize the circuit \emph{beside (fan 3) (id 3)}.

\subsection{Shallow Embeddings and OOP}\label{subsec:shallow}
Shallow embeddings define a language directly through encoding its semantics
using procedural abstraction. In the case of \dsl,
an shallow embedded implementation should conform to the following
types:

\lstinputlisting[linerange=2-2]{./code/shallowCircuit.hs}%APPLY:CIRCUIT_TYPE
\lstinputlisting[linerange=5-7]{./code/shallowCircuit.hs}%APPLY:TYPES
The type \lstinline{Circuit}, representing the semantic domain, is to be filled in with a concrete type according to the semantics.
Suppose that the semantics of \dsl is to calculate the width of a
circuit. The definitions would be:
\lstinputlisting[linerange=11-14]{./code/shallowCircuit.hs}%APPLY:CIRCUIT1
%As shallow embedding is semantics-oriented,
%\bruno{why is ``id'' in bold?}
%\lstinline{Circuit} is just a type
%synonym of \lstinline{Int} and it is already an integer after circuit construction.
%For example, we will immediately get $6$ for \lstinline{beside (fan 3) (id 3)}.
Note that, for this tiny DSL, the Haskell domain is simply
\lstinline{Int}. This domain is a degenerate case of
procedural abstraction, where \lstinline{Int} can be viewed 
as a no argument function. In Haskell, due to laziness, \lstinline{Int}
is a good representation. In a call-by-value language 
a no-argument function \lstinline{() -> Int} would be more
appropriate to deal correctly with potential control-flow 
language constructs. More realistic shallow DSLs, such as parser 
combinators~\cite{leijen01parsec}, tend to have more complex functional domains.

\begin{comment}

A simple rewriting of the previous program is to wrap the result into an
datatype, getting back the value through pattern matching:
\lstinputlisting[linerange=18-25]{./code/shallowCircuit.hs}%APPLY:CIRCUIT2
% Then circuit construction gives back a \lstinline{Circuit2} which contains an
% integer value representing the width and \lstinline{width2} extracts out the value.
% Is this definition still shallow embedding?
%One may be curious whether this variant implementation is still shallow embedding or not. The answer is yes because ... % why it is shallow

\end{comment}

\paragraph{Towards OOP}
A simple, \emph{semantics preserving}, rewriting of the above program is given
below, where a record with a sole field captures the domain and is declared as a \lstinline{newtype}:
\lstinputlisting[linerange=2-5]{./code/Rewrite.hs}%APPLY:CIRCUIT3
%\bruno{width3 should now be width2}
%We no longer need a separate \lstinline{width3} function as the field name is the extractor.
The implementation is still shallow because \lstinline{newtype} does not add any operational
behaviour to the program, and hence the two programs are effectively the
same.  However, having fields makes the program look more like an 
OOP program.

\paragraph{Porting to Scala}
Indeed, we can easily translate the Haskell program into an OO
language like Scala:
\lstinputlisting[linerange=2-15]{../src/width/Circuit.scala}%APPLY:CIRCUIT_SCALA
The record type maps to the trait \lstinline{Circuit} and field
declaration becomes a method declaration.
Each case in the semantic function corresponds to a trait and its parameters become fields of that trait.
And these traits extend \lstinline{Circuit} and implement \lstinline{width}.


This implementation is essentially how we would model \dsl with an OO language in the first
place, following the \interp pattern (which uses \textsc{Composite} pattern to
organize classes). A minor difference is the use of
traits, instead of classes. Using traits instead of
classes enables some additional modularity via multiple (trait-)inheritance.
In summary, shallow embeddings and straightforward OO programming are closely
related.
%It may worth mentioning that deep embedding is closely related to the \textsc{Visitor} pattern.


\section{Interpretations in Shallow Embeddings}

A well-known limitation of shallow embeddings is that they allow only a single
interpretation. Gibbons and Wu worked around this problem by accommodating
multiple interpretations using tuples. However, their encoding needs to modify
the original code. Although a modular solution based on \cite{swierstra2008data}
is also presented, it complicates the encoding dramatically and may prevent pratical use.
OO approach, on the contrary, provides modular yet simple solution of defining
multiple interpretations. This section illustrates how various type of
interpretations can be defined in an OOP way.


\subsection{Multiple Interpretations}
%Defining additional interpretations is not trivial for shallow embedding,
%especially for functional languages.
Suppose that we want to have an additional function that checks whether a circuit is
constructed correctly. Here comes Gibbons and Wu's solution:
\lstinputlisting[linerange=2-9]{./code/NewSemantics.hs}%APPLY:SEMANTICS_HS
which is not modular because
whenever a new interpretation is needed (e.g. \lstinline{wellSized}), the
original code has to be revised -
the arity of the tuple must be incremented and the new interpretation has to be
appended to each case.
%we add the definition of \lstinline{wellSized} by modifying the original code.

Conversely, we can introduce new interpretations in a
modular and intuitive way with an OO language like Scala, as shown in Figure~\ref{code:operation}.
\begin{figure}
\lstinputlisting[linerange=2-15]{../src/wellsized/Circuit.scala}%APPLY:MULTIPLE_SCALA
\caption{Adding new interpretations}
\label{code:operation}
\end{figure}
%Instead of modifying the original code, we define \lstinline{wellSized} modularly.
The encoding relies on three OOP abstraction mechanisms:
\emph{inheritance}, \emph{subtyping} and \emph{type-refinement}.
Specifically, the new \lstinline{Circuit} is a subtype of
\lstinline{base.Circuit} and declares a new method \lstinline{wellSized}.
The hierarchy implements the new \lstinline{Circuit} by inheriting the corresponding class
from \lstinline{base} and
complementing the body of \lstinline{wellSized}.
Also, fields of \lstinline{Beside} are refined with the new \lstinline{Circuit} type
to avoid type mismatch in creating objects.

% Multiple inheritance
We can even define \lstinline{wellSized} separately:
\begin{lstlisting}
trait Circuit { def wellSized: Boolean }
trait Id extends Circuit { ... }
...
\end{lstlisting}
And merge the two hierarchy through \emph{multiple inheritance} for providing
multiple interpretations:
\begin{lstlisting}
trait Circuit
    extends width.Circuit with wellSized.Circuit
trait Id extends Circuit
    with width.Id with wellSized.Circuit
...
\end{lstlisting}

\subsection{Dependent Interpretations}
 \emph{Dependent interpretations} use other interpretations in their definition.
%Such interpretations are non-compositional.
In Haskell a dependent interpretation must be defined together with what it
dependents on and makes no exceptions on modular approaches like~\cite{}.
This prevents a new interpretation that depends on existing
interpretations from being defined modularly.
Fortunately, OO approach does not have such restriction.

Before giving an example of dependent interpretations, we first show how to add new
constructs in OO approach. The extended grammar of \dsl is given below,
which contains two extra constructs \emph{above} and \emph{stretch}:
\setlength{\grammarindent}{5em} % increase separation between LHS/RHS
%\lstinputlisting[linerange=39-40]{./code/shallowCircuit.hs}%APPLY:SYNTAX_TYPES
\begin{grammar}
<circuit> ::= \ldots
\alt `above' <circuit> <circuit>
\alt `stretch' <positive-numbers> <circuit>
\end{grammar}
$above\ c_1\ c_2$ combines two circuits of the same width vertically;
\emph{stretch ns c} inserts additional wires into the circuit \emph{c} by
summing up \emph{ns}.
Figure~\ref{code:variant} shows how to add them to \dsl - simply defining new classes that implement \lstinline{Circuit}.
\begin{figure}
\lstinputlisting[linerange=2-11]{../src/width/Variants.scala}%APPLY:VARIANT_SCALA
\caption{Adding new language constructs}
\label{code:variant}
\end{figure}

From the specifications we can see that not arbitrary circuits can be combined using $above$ and $stretch$.
We hence define the \lstinline{wellSized} method for them to
verify the constraints they imply:
\lstinputlisting[linerange=2-11]{../src/wellsized/Variants.scala}%APPLY:DEPENDENT_SCALA
Definitions of \lstinline{wellSized} for the extended cases make it a dependent
interpretation, as they can not be defined without \lstinline{width}.
Note that \lstinline{width} and \lstinline{wellSized} are defined separately.
Essentially, it is sufficient to define \lstinline{wellSized} with only the signature of \lstinline{width} in \lstinline{Circuit}.

\paragraph{Solving the Expression Problem (EP)} As EDSLs evolve along the time, the
demand for new syntax and new semantics may arise. It would be good if
these features can be introduced modularly. Then the capability of solving the EP
becomes a requirement of the host language.
Figure~\ref{code:operation} and Figure~\ref{code:variant} together illustrate
how elegant the OO solution,
indicating that OO languages are suitable host languages for defining modular EDSLs.

\subsection{Context-sensitive Interpretations}
Unlike \lstinline{width} and \lstinline{wellSized} which can be defined with
only the given circuit, interpretations may need some mutable contexts for definition.
These contexts can be captured by arguments of the method. For example, a
function that collects all the connections between wires inside a circuit would have
the following signature:
\begin{lstlisting}
type Layout = List[List[Tuple2[Int,Int]]]
def tlayout(f: Int => Int): Layout
\end{lstlisting}
where the context \lstinline{f} may vary in recursive calls.
For space reasons, we omit the implementation details.

\input{sections/TypeClass}

\input{sections/Factory}

\input{sections/Casestudy}

\section{Conclusion}
This paper shows how OO programming improves the modularity of shallow EDSLs.
With the procedural abstraction mechanisms provided by OO languages, various types of
interpretations can be defined modularly. We also show that defining
transformations in shallow embeddings is possible.

There are still certain interpretations that can not be modeled modularly using
our approach, including transformations and binary methods.
For transformations, we have to refine their return type in extensions and
supply a new implementation by duplicating the original code.
For binary methods like equality, there is no way to refine the argument type.
Though with advanced type system features of Scala, these can still be defined
modularly~\ref{zenger05independentlyextensible}. However, the encoding would
become complicated.
Also, we can further eliminate some boilerplate through code generation.
For example, similar inheritance relationships are repeated in each constructs
for extensions.


% Bibliography
%\renewcommand{\section}[1]{\oldsec{#1}}

\bibliography{Main}

% History dates
%\received{February 2007}{March 2009}{June 2009}

\end{document}
