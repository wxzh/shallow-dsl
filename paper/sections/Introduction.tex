\section{Introduction}

Since Hudak's seminal paper on Embedded Domain Specific Languages (EDSLs)~\cite{hudak1998modular}, existing
languages (e.g. Haskell) have been used to directly encode
DSLs. Two common approaches to EDSLs are the so-called \emph{shallow}
and \emph{deep} embeddings. The origin of that terminology can be
attributed to Boulton et al.'s work~\cite{Boulton92dsl}. The difference between these
two styles of embeddings is commonly described as follows:

\begin{quote}
\emph{With a deep embedding, terms in the DSL are implemented simply to
construct an abstract syntax tree (AST), which is subsequently
transformed for optimization and traversed for evaluation. With a
shallow embedding, terms in the DSL are implemented directly by
their semantics, bypassing the intermediate AST and its traversal.}\cite{gibbons2014folding}
\end{quote}

%\begin{comment}
%This definition is widely accepted and similar definitions appear in
%many other works~\cite{}. 
%We argue that this definition is vague,
%and often leads to some contradicting claims. 

\noindent Although the above definition is quite reasonable and widely accepted,
it leaves some space to (mis)interpretation. For example it is unclear 
how to classify an EDSL implemented using the {\sc Composite} or {\sc Interpreter} 
patterns in Object-Oriented Programming (OOP). Would this OO approach be
classified as a shallow or deep embedding? We feel there is a rather
fuzzy line here, and the literature allows for both interpretations. Some authors working on
OOP EDSLs ~\cite{rompf2012scala,scherrc2015} consider a {\sc Composite} to be a deep
embedding. Some other authors~\cite{gibbons2014folding,barringer2011tracecontract}
consider implementations using tuples and/or the  {\sc Composite}
pattern to be a shallow embedding.  


To avoid ambiguitity we propose defining shallow embeddings as EDSLs implemented using \emph{procedural abstraction}~\cite{reynolds94proceduralabstraction}. Such
interpretation arises naturally from the domain of shallow EDSLs being
functions, and procedural abstraction being a way to encode
data abstractions using functions. As Cook~\cite{cook09abstraction} argued,
procedural abstraction is also the essence of OOP.
Thus, according to our definition, the implementation of a shallow
EDSL in OOP languages should simply correspond to a standard
object-oriented program. 

The main goal of our paper is to show the close relationship between 
shallow embeddings and OOP, and argue that OOP languages have
advantages for the implementation of shallow embeddings. 
It is perhaps partly due to those advantages that some authors 
have considered  {\sc Composite}-based implementations to be deep
embeddings. However, we argue that the OOP
mechanisms do not change the essence of the implementation, which is
still shallow (i.e. using procedural abstraction). 
To understand how
OOP is helpful for shallow embeddings, we should first look at the
limitations of shallow embeddings commonly found in the literature:
\begin{enumerate}

\item {\bf Single Interpretation} An often stated limitation of
  shallow EDSLs is that they only support one
  interpretation.

\item {\bf No Transformations} Another commonly stated limitation 
of shallow embeddings is the lack of support for transformations,
preventing optimizations.

\end{enumerate}

We show that OOP abstractions, including \emph{
  inheritance}, \emph{subtyping} and \emph{type-refinement}, are
helpful to address those problems. For the first problem, we can
employ a recently proposed design pattern~\cite{eptrivially16}, which provides a simple
solution to the \emph{Expression Problem}~\cite{expPb} in OOP languages. Thus
using just standard OOP mechanisms enables \emph{multiple modular
  interpretations} to co-exist and be combined in shallow embeddings.
For the second problem, we show that transformations can be encoded 
with recursive objects. Interestingly enough, for transformations it
is possible to port back the OO solution to
Haskell.

We make our arguments by taking a recent paper by Gibbons and Wu~\cite{gibbons2014folding},
where procedural abstraction is used in Haskell to model a simple \emph{shallow}
EDSL, and we recode that EDSL in Scala\footnote{Available online: \url{https://github.com/wxzh/shallow-dsl}}.
From the \emph{modularity} point of view the
Scala version has clear advantages over the Haskell version.

\begin{comment}
In summary, our contributions are:

\begin{itemize}

\item {}

\item {\bf Multiple Modular Interpretations for Shallow Embeddings:} 
  We show that with standard OOP mechanisms it is easy to support multiple modular
  interpretations for shallow embeddings.

\item {\bf Transformations for Shallow Embeddings:} We show that
  transformations are encodable with recursive objects. Moreover, this technique
  can be ported back into functional programming as well.

\end{itemize}
\end{comment}

\begin{comment}
If we accept Cook's view on OOP, 
a natural way to distinguish implementations of EDSLs is 
in terms of the data abstraction used to model the language
constructs instead. As Reynold's~\cite{reynolds94proceduralabstraction} suggested there are two
types of data abstraction: procedural abstraction and \emph{user-defined
  types}. It is clear that shallow embeddings use \emph{procedural
  abstraction}: the DSLs are modelled by interpretation
functions. Thus, the other implementation option for EDSLs is to
use \emph{user-defined types}. In Reynolds terminology user-defined
types mean disjoint union types, which are nowadays commonly available
in modern languages as \emph{algebraic datatypes}. Disjoint union
types can also be emulated in OOP using the {\sc Visitor} pattern. 

A distinction based on data abstraction is more precise and provides a
remedy for possible misinterpretation. An EDSL implemented with
algebraic datatypes falls into the category of user-defined types
(deep embedding), while a {\sc Composite}-based OO implementation falls
under procedural abstraction (shallow embedding). 
\end{comment}

%At least some authors~\cite{} seem to implicitly
%agree with the latter interpretation.

%The {\sc Composite} or {\sc Interpreter}
%patterns are normally accepted to provide a way to encode ASTs. Thus, 
%one possible interpretation is that \emph{according to definition of deep embedding
%  above, the OO approach classifies as a deep
%  embedding}. 

\begin{comment}
For example, in their work
on EDSLs~\cite{}, Gibbons and Wu claim that deep embeddings (which
encode ASTs using algebraic datatypes in Haskell) allow adding new DSL
interpretations easily, but they make adding new language constructs
difficult. In contrast Gibbons and Wu claim that shallow embeddings
have dual modularity properties: new cases are easy to add, but new
interpretations are hard.  However what if, instead of using Haskell
and algebraic datatypes, one uses an OO language to encode an AST, for
example with the {\sc Composite} pattern.  Would this OO approach be
classified as a shallow or deep embedding? We believe arguments can be
made both ways. Since the {\sc Composite}
pattern is normally accepted to be a way to encode ASTs, it would be
reasonable to say that \emph{according to definition of deep embedding
  above, the OO approach classifies as a deep
  embedding}. Unfortunatelly this interpretation could be problematic.
As the Expression Problem~\cite{} tell us,
in the OO approach adding new language constructs is easy, but adding
interpretations is hard. Thus this would contradict Gibbons and Wu's
claims, since we have an AST representation (i.e. a deep embedding)
with the modularity properties of shallow embeddings.

We believe that the core of problem is that ASTs can be represented in
multiple ways. In particular, it is well know that functions alone are
enough to encode datastructures such as ASTs (via Church
encodings~\cite{}).  Distinguishing deep and shallow embeddings based
solely on whether a ``real'' datastructure is being used or not is
misleading.  Moreover, it gives the impression that shallow embeddings
are significantly less expressive than deep embeddings, because they
do not have access to the datastructure.
Gibbons and Wu themselves feel uneasy with the definition of shallow 
embeddings when they say:
``\emph{So it turns out that the syntax of the DSL is not really as ephemeral
in a shallow embedding as Boulton's choice of terms suggests.}''
\end{comment}