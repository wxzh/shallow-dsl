\section{Case Study}

Weixin and Haoyuan.

Talk about the 3 case studies

Talk about Pretty Printer case study last and discuss issues with
extensible transformations.

\subsection{Scans}
% statistics, pictures, implementation
Section~\ref{} is based on our first case study, hence we discuss it briefly here.
In the case study, we strictly follows the flow of the paper and implement all kinds of
interpretations modularly, including \texttt{depth} and \texttt{layout} that are
not discussed in Section~\ref{}.

Starting from a data structure that describes the five constructs without any interpretations, we gradually add
\texttt{depth}, \texttt{width}, \texttt{wellSized}, \texttt{layout} and
\texttt{tlayout} through extending the family.

To make our DSL more user-friendly, we defined some wrappers for constructing
circuits conveniently. Afterwards, we can construct a circuit this way:
\begin{lstlisting}
Circuit circuit =
  fan(2).beside(fan(2))
  .above(fan(2).stretch(2,2))
  .above(id(1).beside(fan(2)).beside(id(1)));
\end{lstlisting}
For primitives like \texttt{fan}, we define a static method and for binary
combinators like \texttt{above}, we define a default method on the \texttt{Circuit}.

Additionally, we extend the language with a \texttt{draw} method which renders a
circuit using Java Swing.
Calling \texttt{circuit.draw()} on the \texttt{circuit} we just defined will
display what Figure~\ref{} shows.

% show Java swing output
\begin{figure}
\end{figure}

\subsection{A Prettier Printer}

This case study refactors the Haskell code from \cite{Wadler98aprettier}
which uses deep embeddings to implement a functional pretty printer library with high efficiency. In the original code, two
data structures are defined for documents:
\begin{lstlisting}[keywords={data,Int,String}]
data DOC = NIL
            | DOC :<> DOC
            | NEST Int DOC
            | TEXT String
            | LINE
            | DOC :<> DOC
data Doc = Nil
            | String `Text` Doc
            | Int `Line` Doc
\end{lstlisting}
Here they are encoded with shallow DSLs, and packaged into
two base families with \textsf{@Family} annotation applied. These two families \textsf{Family\_Doc} and
\textsf{Family\_Document} correspond to data types \textsf{Doc} and \textsf{DOC} respectively.

\begin{lstlisting}
@Family interface Family_Doc {
	interface Doc {}
	interface Nil extends Doc {}
	interface Text extends Doc {
		String _s(); Doc _d();
	}
	interface Line extends Doc { int _i(); Doc _d(); }
}

@Family interface Family_Document {
	interface Document {}
	interface DNil extends Document {}
	interface DConcat extends Document {
		Document _d1(); Document _d2();
	}
	interface DNest extends Document {
		int _i(); Document _d();
	}
	interface DText extends Document { String _s(); }
	interface DLine extends Document {}
	interface DUnion extends Document {
		Document _d1(); Document _d2();
	}
}
\end{lstlisting}

With the help of family polymorphism and \textsf{@Family}, we can easily add new operations and integrate them
in child families. Below is an example of encoding two original functions \textsf{layout} and \textsf{fits} in
the shallow embeddings. Note that with the annotation processing, verbose code for building inheritance relations is
automatically generated.
\begin{lstlisting}
@Family interface Family_Doc_LayoutFits extends Family_Doc {
	interface Doc {
		String layout(); boolean fits(int w);
	}
	interface Nil {
		default String layout() { return ""; }
		default boolean fits(int w) {
			return w >= 0;
		}
	}
	...
\end{lstlisting}
Furthermore, the function \textsf{be} needs refactoring before it is encoded in shallow embeddings.
The original \textsf{be} is defined as follows:
\begin{lstlisting}
be w k [] = Nil
be w k ((i,NIL):z) = be w k z
be w k ((i,x :<> y):z) = be w k ((i,x):(i,y):z)
be w k ((i,NEST j x):z) = be w k ((i+j,x):z)
be w k ((i,TEXT s):z) = s `Text` be w (k+length s) z
be w k ((i,LINE):z) = i `Line` be w i z
be w k ((i,x :<|> y):z) = better w k (be w k ((i,x):z)) (be w k ((i,y):z))
\end{lstlisting}
We refactor it by introducing a helper function \textsf{beaux}:
\begin{lstlisting}
beaux w k i NIL z = be w k z
beaux w k i (x :<> y) z = be w k ((i,x):(i,y):z)
beaux w k i (NEST j x) z = be w k ((i+j,x):z)
beaux w k i (TEXT s) z = s `Text` be w (k+length s) z
beaux w k i LINE z = i `Line` be w i z
beaux w k i (x :<|> y) z = better w k (be w k ((i,x):z)) (be w k ((i,y):z))

be w k [] = Nil
be w k ((i,x):z) = beaux w k i x z
\end{lstlisting}
In this case, \textsf{beaux} can be defined as an operation method \lstinline{beaux(int, int, int, List<Pair<int, Document>>)} inside the type \textsf{Document}, and
its pattern matching corresponds to the default implementations in different member types (like \textsf{DNil}, \textsf{DConcat}, etc). On the other hand, \textsf{be} is implemented as a static method
\lstinline{be(int, int, List<Pair<int, Document>>)} outside the member types.

Finally we finished the refactoring of the pretty printer library using a set of extensible families with shallow DSLs and operations, and our
\textsf{@Family} annotation. However there is a special operation called \textsf{group}, which works like a transformation:
\begin{lstlisting}
@Family interface Family_Document_Flatten extends Family_Document {
	interface Document {
		Document flatten();
		default Document group() {
			return DUnion.of(this.flatten(), this);
		}
	}
	...
}
\end{lstlisting}
The method \textsf{group} calls the constructor method from \textsf{DUnion}, whereas such a member type will be updated in the
child families, together with its constructors. To ensure type safety we are unable to reuse the code but just copy it and paste
to the child families, and hence code duplication is introduced by the use of factory methods. The matter of extensible
transformations will be further discussed in the next section.




\subsection{Diagrams}
\emph{Diagrams}~\cite{} is a powerful, flexible, declarative Haskell DSL for creating vector graphics.
This case study implements a simplified version of \emph{Diagrams} from
Gibbons's lecture notes ~\cite{}.
The DSL consists of three simpler sub-languages for describing shapes, colors, styles.
% Haskell code
\begin{lstlisting}[language=haskell]
data Shape
  = Rectangle Double Double
  | Ellipse Double Double
  | Triangle Double

type StyleSheet = [Styling]
data Styling
  = FillColour Col
  | StrokeColour Col
  | StrokeWidth Double

data Col = Red | Blue | Bisque | Black | Green | Yellow | Brown

data Picture
  = Place StyleSheet Shape
  | Above Picture Picture
  | Beside Picture Picture
\end{lstlisting}

\paragraph{Shapes}
There are three primitive shapes - rectangle, ellipse and triangle.

\begin{lstlisting}
interface Shape {}
interface Rectangle extends Shape {
    double _x(); double _y();
}
interface Ellipse extends Shape {
    double _rx(); double _ry();
}
interface Triangle extends Shape {
    double _l();
}
\end{lstlisting}

\paragraph{Styles}
A shape can be decorated with drawing styles. A \texttt{StyleSheet} contains a
list of \texttt{Stylings} where fill color, stroke color or stroke width can be specified.
\begin{lstlisting}
interface StyleSheet {
    List<? extends Styling> _stylings();
}
interface Styling {}
interface FillColor extends Styling {
    Col _color();
}
interface StrokeColor extends Styling {
    Col _color();
}
interface StrokeWidth extends Styling {
    double _width();
}
\end{lstlisting}
\texttt{Col} is also a data structure that describes the supported colors.
\begin{lstlisting}
interface Col {}
interface Red extends Col {}
interface Blue extends Col {}
interface Green extends Col {}
interface Yellow extends Col {}
interface Bisque extends Col {}
interface Black extends Col {}
\end{lstlisting}

\paragraph{Pictures}
\begin{lstlisting}
interface Picture {}
interface Place extends Picture {
    Shape _s(); StyleSheet _ss();
}
interface Above extends Picture {
    Picture _p1(); Picture _p2();
}
interface Beside extends Picture {
    Picture _p1(); Picture _p2();
}
\end{lstlisting}

Provided with a \texttt{StyleSheet}, a shape can be lifted to a picture using
\texttt{Place}. And pictures can be merged using \texttt{Above} or
\texttt{Beside}: the former combines two pictures vertically whereas the latter
combines two pictures horizontally.

With these three main data structure, we can draw a picture.

\subsubsection{Rendering}
We render our fancy picture using scalable vector graphics (SVG) as our backend.
To do so, we need to flatten the picture which is recursively constructed.

SVG is based on XML and hence we define to model a XML file.

\paragraph{Transformations}

\subsubsection{Extensions}
\cite{} contains exercises about extensions on the language, e.g.
adding new shapes. The deep embedding approach in Haskell requires modifications on the ADTs
whereas our approach can modularly introduce new language constructs.
.
Each sub-language retains independent extensibility.
For example, we can add a \texttt{show} method for the purpose of debugging.
