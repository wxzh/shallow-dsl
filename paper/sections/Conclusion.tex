\section{Discussion and Conclusion}
This paper shows how OO programming improves the modularity of shallow EDSLs.
With the procedural abstraction mechanisms provided by OO languages, various types of
interpretations can be defined modularly. We also show that defining
transformations in shallow embeddings is possible.

Existing work showed that shallow embeddings yield flexible and concise EDSLs, while deep embedding makes
it easy to define optimizations.
There is a lot of work~\cite{svenningsson2012combining,
  Jovanovic:2014:YCD:2658761.2658771, scherr2014implicit} trying to blend these two
approaches to enjoy benefits from both.
They typically encode the surface language with a shallow embedding and
then generate or translate to a deep embedded version for allowing optimizations.
%Hofer and Ostermann~\cite{hofer2010modular} propose to provide both embedding through implementing internal and external visitor at the same time so that clients can choose for a particular interpretation;
The OO approach we present retains the simplicity of shallow embedding while
making it possible to implement optimizations. It would be interesting
to conduct larger case studies to assess whether the techniques
presented here are enough to avoiding deep embeddings for various DSLs
in the literature.

One limitation of the techniques presented here is that
transformations require code duplication in extensions.  For
transformations, we have to refine their return type in extensions and
supply a new implementation by duplicating the original code.  How to
define these transformations modularly, and without code duplication
is a possible line of future work.  Also, it is still tedious to
express the inheritance relationships in extensions, especially when
multiple inheritance is used. Another line of future work is to use
some meta-programming mechanisms to eliminate such boilerplate.
