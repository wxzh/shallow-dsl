\section{Interpretations in Shallow Embeddings}

\begin{comment}
Weixin writes this one.

Go over jeremy's examples, maybe having only 3 diagram 
constructs instead of 5 for space reasons.

Use five constructs to show extensibility.

Think of how to introduce our tool? Using the Jeremy's examples? 
or introducing before with some other examples?

\end{comment}

A well-known limitation of shallow embeddings is that they allow only a single
interpretation. Gibbons and Wu worked around this problem by accommodating
multiple interpretations using tuples and summarized how to define various types
of interpretations with a shallow embedding.
This section illustrates the OOP way of defining interpretations. Different from
their workaround, our approach is modular extensible.

\subsection{Multiple Interpretations}
%Defining additional interpretations is not trivial for shallow embedding,
%especially for functional languages.
Suppose that we want to have an additional function that checks whether a circuit is
constructed correctly. Here comes the Haskell solution:
\lstinputlisting[linerange=2-9]{./code/NewSemantics.hs}%APPLY:SEMANTICS_HS
which is not modular because we add the definition of \lstinline{wellSized} by
modifying the original code. And whenever a new interpretation comes,
the arity of the tuple must be incremented and the new interpretation has to be
appended to each case.
Though Gibbons and Wu presented an modular way of defining multiple
interpretations in Haskell based on
\cite{swierstra2008data}. But such solution changes the encoding style
dramatically and may be too complex to use in practice.

Conversely, we can define multiple interpretations in a
modular and intuitive way with an OO language like Scala:
\lstinputlisting[linerange=2-15]{../src/wellsized/Circuit.scala}%APPLY:MULTIPLE_SCALA
Instead of modifying the original code, we define \lstinline{wellSized} modularly.
The encoding makes use of three OOP abstraction mechanisms:
\emph{inheritance}, \emph{subtyping} and \emph{type-refinement}.
Specifically, the new \lstinline{Circuit} is a subtype of
\lstinline{base.Circuit} and declares a new method \lstinline{wellSized}.
The hierarchy implements the new \lstinline{Circuit} by inheriting the corresponding class
from \lstinline{base} and
complementing the body of \lstinline{wellSized}.
Also, fields of \lstinline{Beside} are refined with the new \lstinline{Circuit}
type to avoid type mismatch in creating objects.


\subsection{Dependent Interpretations}
 \emph{Dependent interpretations} can not be defined alone which use other
 interpretations in their definition.
%Such interpretations are non-compositional.
In Haskell a dependent interpretation must be defined together with what it
dependents on and makes no exceptions on modular approaches like~\cite{}.
This prevents a new interpretation that depends on existing
interpretations from being defined modularly.
Fortunately, OO approach can still modularize such interpretations.

Before giving an example of dependent interpretations, we first show how to add new
constructs. The new constructs are: \emph{stretch ns c} which inserts additional wires into the circuit \emph{c} by
summing up \emph{ns} and $above\ c_1\ c_2$ which combines two circuits of the same width vertically.
Their types are given below:
\lstinputlisting[linerange=39-40]{./code/shallowCircuit.hs}%APPLY:SYNTAX_TYPES

Adding new constructs is easy - just defining new classes that implement \lstinline{Circuit}:
\lstinputlisting[linerange=19-30]{../src/wellsized/Circuit.scala}%APPLY:VARIANT
Not arbitrary circuits can be combined using $above$ and $stretch$
as stated in their specifications.
We hence define the \texttt{wellSized} method for these two new constructs to
verify the constraints they imply:

Definitions of \texttt{wellSized} for the extended cases make it a dependent interpretation, as they
rely on another method defined on the \texttt{Circuit}, \texttt{width}, for calculating the width of a circuit.
Most importantly, our approach do not force dependent interpretation to be
defined together with what they depend on. In the case of \texttt{width} and
\texttt{wellSized}, they can be defined in separate families as long as the
signature of \texttt{width} is exposed to the definition of \texttt{wellSized}.

\paragraph{Solving Expression Problem} The code above shows how to
define new variant, combining with

\subsection{Context-sensitive Interpretations}
by declaring contexts as arguments of the method.
For space reasons, we omit the.