\subsection{A Prettier Printer}

This case study refactors the Haskell code from \cite{Wadler98aprettier}
which uses deep embeddings to implement a functional pretty printer library with high efficiency. In the original code, two
data structures are defined for documents:
\begin{lstlisting}[keywords={data,Int,String}]
data DOC = NIL
            | DOC :<> DOC
            | NEST Int DOC
            | TEXT String
            | LINE
            | DOC :<> DOC
data Doc = Nil
            | String `Text` Doc
            | Int `Line` Doc
\end{lstlisting}
Here they are encoded with shallow DSLs, and packaged into
two base families with \textsf{@Family} annotation applied. These two families \textsf{Family\_Doc} and
\textsf{Family\_Document} correspond to data types \textsf{Doc} and \textsf{DOC} respectively.

\begin{lstlisting}
@Family interface Family_Doc {
	interface Doc {}
	interface Nil extends Doc {}
	interface Text extends Doc {
		String _s(); Doc _d();
	}
	interface Line extends Doc { int _i(); Doc _d(); }
}

@Family interface Family_Document {
	interface Document {}
	interface DNil extends Document {}
	interface DConcat extends Document {
		Document _d1(); Document _d2();
	}
	interface DNest extends Document {
		int _i(); Document _d();
	}
	interface DText extends Document { String _s(); }
	interface DLine extends Document {}
	interface DUnion extends Document {
		Document _d1(); Document _d2();
	}
}
\end{lstlisting}

With the help of family polymorphism and \textsf{@Family}, we can easily add new operations and integrate them
in child families. Below is an example of encoding two original functions \textsf{layout} and \textsf{fits} in
the shallow embeddings. Note that with the annotation processing, verbose code for building inheritance relations is
automatically generated.
\begin{lstlisting}
@Family interface Family_Doc_LayoutFits extends Family_Doc {
	interface Doc {
		String layout(); boolean fits(int w);
	}
	interface Nil {
		default String layout() { return ""; }
		default boolean fits(int w) {
			return w >= 0;
		}
	}
	...
\end{lstlisting}
Furthermore, the function \textsf{be} needs refactoring before it is encoded in shallow embeddings.
The original \textsf{be} is defined as follows:
\begin{lstlisting}
be w k [] = Nil
be w k ((i,NIL):z) = be w k z
be w k ((i,x :<> y):z) = be w k ((i,x):(i,y):z)
be w k ((i,NEST j x):z) = be w k ((i+j,x):z)
be w k ((i,TEXT s):z) = s `Text` be w (k+length s) z
be w k ((i,LINE):z) = i `Line` be w i z
be w k ((i,x :<|> y):z) = better w k (be w k ((i,x):z)) (be w k ((i,y):z))
\end{lstlisting}
We refactor it by introducing a helper function \textsf{beaux}:
\begin{lstlisting}
beaux w k i NIL z = be w k z
beaux w k i (x :<> y) z = be w k ((i,x):(i,y):z)
beaux w k i (NEST j x) z = be w k ((i+j,x):z)
beaux w k i (TEXT s) z = s `Text` be w (k+length s) z
beaux w k i LINE z = i `Line` be w i z
beaux w k i (x :<|> y) z = better w k (be w k ((i,x):z)) (be w k ((i,y):z))

be w k [] = Nil
be w k ((i,x):z) = beaux w k i x z
\end{lstlisting}
In this case, \textsf{beaux} can be defined as an operation method \lstinline{beaux(int, int, int, List<Pair<int, Document>>)} inside the type \textsf{Document}, and
its pattern matching corresponds to the default implementations in different member types (like \textsf{DNil}, \textsf{DConcat}, etc). On the other hand, \textsf{be} is implemented as a static method
\lstinline{be(int, int, List<Pair<int, Document>>)} outside the member types.

Finally we finished the refactoring of the pretty printer library using a set of extensible families with shallow DSLs and operations, and our
\textsf{@Family} annotation. However there is a special operation called \textsf{group}, which works like a transformation:
\begin{lstlisting}
@Family interface Family_Document_Flatten extends Family_Document {
	interface Document {
		Document flatten();
		default Document group() {
			return DUnion.of(this.flatten(), this);
		}
	}
	...
}
\end{lstlisting}
The method \textsf{group} calls the constructor method from \textsf{DUnion}, whereas such a member type will be updated in the
child families, together with its constructors. To ensure type safety we are unable to reuse the code but just copy it and paste
to the child families, and hence code duplication is introduced by the use of factory methods. The matter of extensible
transformations will be further discussed in the next section.
