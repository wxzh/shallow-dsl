\section{Related Work}

Weixin and Haoyuan.

\paragraph{Deep and Shallow Embeddings}

\paragraph{Modularity of DSLs?}
Data types a la Carte; Finally Tagless;
Object Algebras

\paragraph{Family Polymorphism}

There has been a lot of related work on family polymorphism[1], including various lightweight encodings[2,3,4,5]. In the original
paper[1], nested classes are represented as attributes of an object, which involves a dependent type system. In [4], Saito and Igarashi
proposed a lightweight variant of family polymorphism, which uses classes rather than objects to represent families. Later work in [4]
proposes a simple extension to Featherweight Generic Java[6], introducing self-type variables. There are also some existing languages,
like Scala, which supports symmetric mixin compositions and self-type annotations, hence can provide better support for encoding family
polymorphism. Our lightweight encoding relies entirely on the existing Java language without language extensions, and certainly some features from
the original paper are sacrificed, including the mismatching problem of recursive class definitions (also known as binary methods).

On the other hand, our approach uses nested interfaces to build the relationship among families and members, which is different from
path-dependent types in [1]. The work in [8] inspires us with the concept of higher-order hierarchies, and hence our \textsf{@Family} annotation
provides support for nested families. Furthermore, our lightweight encoding of family polymorphism is polished with the help of annotation processing
to generate constructor methods automatically.

[1] family polymorphism
[2] Lightweight scalable components
[3] The essence of lightweight family polymorphism
[4] Lightweight family polymorphism
[5] Lightweight dependent classes
[6] Featherweight Java
[7] On binary method
[8] Higher-order hierarchies

\paragraph{ThisType}

\paragraph{Multiple Inheritance}
