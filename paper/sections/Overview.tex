\section{Evolution on DSLs with \name}

\begin{comment}
Weixin writes this one.

Go over jeremy's examples, maybe having only 3 diagram 
constructs instead of 5 for space reasons.

Use five constructs to show extensibility.

Think of how to introduce our tool? Using the Jeremy's examples? 
or introducing before with some other examples?

\end{comment}

%As language evolves, the need of new syntax often arises.
As DSLs evolve along the time, the demand for new syntax and new semantics may arise.
It would be good if we can introduce these new features to the DSL \emph{modularly}.
%This is actually a hard problem, known as the Expression Problem (EP)~\cite{wadler}, which
%This, however, requires the host language equipped with two dimensions of extensibility.
For shallow embedding, adding new syntax is trivial - just adding new cases in
the semantic function. On the other hand, adding new semantics becomes hard as it allows
only one definition for each case. Gibbons and Wu summarized how to define
various type of semantic functions (interpretations) in shallow embedding by
applying \emph{fold theory}. However, the encoding is based on tuples, which is
not modular extensible. Though they also presented modular solution based on
datatype a lar carte\cite{}. But such solution changes the encoding style
dramatically and may be too complex to use practically.
In contrast, the OO approach allows all kinds of
interpretations to be defined in an modular intuitive way.
In this section, we will show how various types of interpretations can be
defined using the OO way with the help of \name. Along the way,
some features of \name will be introduced, which simplify our implementation significantly.

%argue that from extensibility perspective, OO languages are
%better candidates as host languages than FP languages.
%To demonstrate this, we try to extend \dsl with new syntax and new semantics in
%the setting of shallow embedding. Adding new syntax is both easy for both FP
%and OOP. Adding new semantics, however, is hard in FP. Although it is possible
%in OOP, the solution requires some boilerplate code. We hence developed \name
%for defining modular extensions easily.

\subsection{Initial System}
Recall the \dsl language discussed in Section~\ref{}.
We redefine it with \name:

\lstinputlisting[linerange=19-35]{../src/paper/sec3/Circuit.java}%APPLY:INIT

There are three main changes:
1) All the original implementations are moved inside an interface \texttt{Init}; and
2) The constructor classes are now interfaces for the purpose of employing
multiple interface inheritance of Java;
3) Fields of the classes become getter methods (starting with an underscore) so that their types can be refined for retaining extensibility~\cite{}.

% introduce of method
As classes are now interfaces, how can we create objects on them?
\name automatically generates a \emph{static} \texttt{of} method for each
class-like interface, i.e. an interface that contains only getters, for creating objects.
For instance, the \texttt{of} method for \texttt{Beside} is:
\begin{lstlisting}
static Beside of(Circuit c1, Circuit c2) {
  return new Beside() {
    Circuit _c1 = c1; Circuit _c2 = c2;
    public Circuit _c1() { return c1; }
    public Circuit _c2() { return c2; }
  }
}
\end{lstlisting}
Inside the \texttt{of} method, an anonymous class that implements
\texttt{Beside} is created and its instance is returned.

\subsection{Multiple Interpretations}
Defining additional interpretations is not so trivial for shallow embedding,
especially for functional languages.
Suppose that we want to have a function that checks whether a circuit is constructed correctly.

Here comes the solution from Gibbons and Wu.
\lstinputlisting[language=haskell,linerange=2-8]{./code/NewSemantics.hs}%APPLY:SEMANTICS_HS
where tuples are used to aggregate multiple interpretations in one definition.
The way to introduce \texttt{desugar} in previous section is similar to this
approach, where we use records instead.
However, neither tuples nor records are extensible - whenever we need a new
interpretation, we have to revise the original code, either expanding the arith of the
tuple or inserting a new field in the record.

Conversely, the OO approach allows us to define multiple interpretations in a
\emph{modular} way through extending existing code:
\lstinputlisting[linerange=39-52]{../src/paper/sec3/Circuit.java}%APPLY:MULTIPLE
We expand the semantics in an extended family \texttt{Multiple}.
Inside \texttt{Multiple}, the \texttt{Circuit} is re-declared
with a new method \texttt{wellSized} and the \texttt{Circuit} hierarchy
implements the \texttt{wellSized} method.
Through extending the \texttt{Init} family, the \texttt{width} method is
inherited without duplicating its definition. This way we modularly extend the
semantics.

Note that we do not explicitly write down \texttt{extends} clauses in the
extended family.
Instead, they are inferred by \name according to their
names and the inheritance relationship between families.
After instrumentation done by \name, the definitions of \texttt{Circuit} and \texttt{Beside}
would be:

\begin{lstlisting}
interface Circuit extends Init.Circuit {
  ... // wellSized declaration
}
interface Beside extends Circuit, Init.Beside {
  Circuit _c1(); Circuit _c2(); // type refinements
  ... // wellSized implementation
}
\end{lstlisting}
Interfaces from base families of the same name will be put after \texttt{extends} clause and the
interface hierarchy of \texttt{Circuit} is preserved in the extended family.
Moreover, \name re-declare getters in the extended \texttt{Beside} so that their return types are refined to be the new \texttt{Circuit}.

\subsection{Dependent Interpretations}
Interpretations which depend on other interpretations are called \emph{dependent
  interpretation}. %They are hard to express modularly in traditional shallow embedding.
Before we introduce an example of such interpretations, we first add two new
constructs to our language:
\emph{stretch ns c} inserts additional wires into the circuit \emph{c} by
summing up \emph{ns} and $above\ c_1\ c_2$ combines two circuits of the same width vertically.
Their types are given below:
\lstinputlisting[language=haskell,linerange=40-41]{./code/shallowCircuit.hs}%APPLY:SYNTAX_TYPES


%\lstinputlisting[linerange=56-69]{../src/paper/sec3/Circuit.java}%APPLY:SYNTAX
%With these new constructs, more complex circuits can be constructed.
%For example, Figure~\ref{} shows the circuit constructed by \emph{stretch
%  [2,2,2] (fan 3) `beside` fan 1}.

Adding new constructs in \name is easy: just we declare new class-like
interfaces in an extended family:
\lstinputlisting[linerange=56-69]{../src/paper/sec3/Circuit.java}%APPLY:SYNTAX
Only definitions of the new constructs \texttt{Above} and \texttt{Stretch} are given.
Their peer constructs and \texttt{Circuit} are implicitly re-declared by \name,
e.g. \texttt{Fan}:

\begin{lstlisting}
interface Fan extends Init.Fan {}
\end{lstlisting}

Not arbitrary circuits can be combined using \texttt{Above} and \texttt{Stretch}.
We hence define the \texttt{wellSized} method for these two new constructs to
verify the constraints they imply:
\lstinputlisting[linerange=73-85]{../src/paper/sec3/Circuit.java}%APPLY:DEPENDENT

Definitions of \texttt{wellSized} for the extended cases make it a dependent interpretation, as they
rely on another method defined on the \texttt{Circuit}, \texttt{width}, for calculating the width of a circuit.

Here, we show yet another appealing feature of \name - interpretation combination.
The \texttt{Dependent} family combines two base families

merge interpretations defined for the same data structure together.
%Since \texttt{Circuit} in \texttt{Dependent} family extends the one from
%\texttt{Init}, we can simply call \texttt{width} method on a
%\texttt{Circuit} object when the width of a circuit is needed in the definition
%of \texttt{wellSized}.

\subsection{Context-sensitive Interpretations}
% add tlayout
Connections between vertical wires of a circuit can be modeled by dividing the
circuit into different layers, with local connections only go rightwards from
one layer to the next. Each layer is a collection of pairs $(i,j)$ denoting a
connection from wire $i$ on this layer to wire $j$ on the next. The circuit shown in
Figure\ref{} has the following connections:

$$[[(0,1),(2,3)],[(1,3)],[(1,2)]]$$

The above shows that the circuit has 3 layers: the first layer has connections
from the first wire to the second, and from the second to the third; the second
layer has only one connection from the first wire to the third wire; while the
third layer has a connection from the first wire to the second.

The method \texttt{tlayout} deduce the connections of a circuit. It additionally
takes a parameter \texttt{}.

\lstinputlisting[linerange=107-139]{../src/paper/sec3/Circuit.java}%APPLY:CONTEXT_SENSITIVE

\texttt{tlayout} takes a transformation over \texttt{IntUnaryOperator} is considered as the context which might be changed.
The accumulating argument \texttt{f} is a transformation function over the indices of wires in a circuit.

Some definitions of helper methods are omitted for space reason.
Their functionalities are: \texttt{lzw} stands for 'long zip with', which takes two lists and a binary operator for
combining elements of the same position and if one list is shorter then the
remaining elements of the other are appended to the result list; \texttt{concat}
concatenates two lists with the first tail connected the second head;
The indices of wires are calculated by accumulating the sum of the list \texttt{\_ns} using the \texttt{scanl1}.

% in \texttt{CtxSensitive}

\begin{comment}
\subsection{Interpretation Combination}
One appealing feature that \name approach provides is that we can modularly
merge interpretations defined for the same data structure together.
This is done through from different families
together using multiple interface inheritance.

For example, we can merge the interpretations we have defined so far in this way:

\lstinputlisting[linerange=170-172]{../src/paper/sec3/Circuit.java}%APPLY:MERGE
Simply defining a new family \texttt{Merge} that extends other families we want
to combine. \name will implicitly re-declare all the interfaces and inject the
inheritance relationships. For example, the implicit redeclaration of
\texttt{Circuit} looks like this:

\begin{lstlisting}
interface Circuit extends Multiple.Circuit, Dependent.Circuit, CtxSensitive.Circuit {}
\end{lstlisting}

So do other interfaces.

\subsection{Parametrized interpretations}
\subsection{Implicitly Parametrized interpretations}
\subsection{Modular Interpretations}

\subsection{Adding New Syntax}
Shallow embedding makes it easy to add new syntax, both in FP and OOP.
Suppose that we would like to add two new constructs to the language:

\lstinputlisting[language=haskell,linerange=40-41]{./code/shallowCircuit.hs}%APPLY:SYNTAX_TYPES
\emph{stretch ns c} inserts additional wires into the circuit \emph{c} by
summing up \emph{ns} and $above c_1 c_2$ combines two circuits of the same width vertically.
With these new constructs, more complex circuits can be constructed.
For example, Figure~\ref{} shows the circuit constructed by \emph{stretch [2,2,2] (fan 3) `beside` fan 1}.

To accomplish this task in Haskell, we can simply define two more cases,
\texttt{stretch} and \texttt{above}, for the semantic function:
\lstinputlisting[language=haskell,linerange=45-46]{./code/shallowCircuit.hs}%APPLY:SYNTAX_HS
Similarly, defining two new interfaces
\texttt{Stretch} and \texttt{Above} that both extend \texttt{Circuit} and
implement the \texttt{width} method is all we need to do in Java:

\lstinputlisting[linerange=56-69]{../src/paper/sec3/Circuit.java}%APPLY:SYNTAX

\subsection{Adding New Semantics}
The new combinators, however, can not apply to arbitrary circuits - they have
some invariants in their definitions.
To make sure that a circuit is constructed correctly, we need to expand the
semantics of the language for doing such checks.

Adding new semantics, however, becomes hard for shallow embedding.
Gibbons and Wu worked around this problem in the following way:
\lstinputlisting[language=haskell,linerange=2-8]{./code/NewSemantics.hs}%APPLY:SEMANTICS_HS
which is similar to how we add \texttt{desugar} to the implementation in Section~\ref{}.
The difference is that they use a tuple instead of a record to merge semantic
domains, then define the two semantic functions simultaneously for each
case, and split the semantic functions through projections on the tuple in the end.
This solution, however, modifies existing code, breaking the requirement of EP.

Conversely, the support of covariant type-refinements and inheritance for OO
languages allows us to add new semantics in a \emph{modular} way:
%Different from records or tuples, interfaces are extensible.
\lstinputlisting[linerange=47-75]{../src/paper/sec3/Circuit.java}%APPLY:SEMANTICS

Interface \texttt{CircuitWS} extends the original interface and declare a new
semantic function \texttt{wellSized} inside.
Then all existing cases should extend both their corresponding original
implementation and \texttt{CircuitWS} and implement the new method
\texttt{wellSized}. Also, all the occurrences of \texttt{Circuit} are
refined as \texttt{CircuitWS} so that we can call \texttt{wellSized} on inner circuits returned by getters.
As Java does not support type-refinements on fields, we hence implement these
constructs as interfaces rather than classes.

\subsection{\name's Support}
There exists some boilerplate for the Java solution presented above:
\begin{itemize}
  \item Interfaces that represent language constructs should be instantiated for
    creating objects;
  \item Type-refinements should be done manually;
  \item Similar inheritance relationships have to be repeatedly stated for all interfaces.
\end{itemize}
Worse still, programmers would not get warned if they forget to extend all the
constructs with new semantics.

\name addresses all these problems through embracing family polymorphism and code instrument.
By using \name, we can refactor the extensions in the way shown in Figure\ref{}.
\lstinputlisting[linerange=46-48]{../src/paper/sec3/Circuit.java}%APPLY:FAMILY
\lstinputlisting[linerange=76-78]{../src/paper/sec3/Circuit.java}%APPLY:FAMILY_SYNTAX
\lstinputlisting[linerange=97-122]{../src/paper/sec3/Circuit.java}%APPLY:FAMILY_SEMANTICS

Note that initially all the definitions are put inside a \texttt{Family} interface.
Extensions are defined inside another interface that extends base families.
For syntax extension, we just move the definitions of new constructs into
\texttt{SyntaxExt} interface.
Semantic extension is more interesting, let us look at it in detail.
Since all the definitions are nested interfaces, their names are local and can
be reused in other families. The inheritance relationships is stated only once at family level, then \name can infer the inheritance relationships for members inside the family according to their names. Also, there is no need to
manually refine the return type, as it is done by \name. Moreover, a static \texttt{of}
method would be generated for class-like interfaces, serving as the constructor
for the interface. For example, the \texttt{Beside} inside \texttt{Semantics} after code instrumentation will look like this:

\lstinputlisting[linerange=177-189]{../src/paper/sec3/Circuit.java}%APPLY:INSTRUMENT
% TODO: not implemented yet
Family polymorphism on only gives more safety on the client code but also helps
language implementers catch bug early.
For example, if one forgets to implement, say cases from \texttt{Family}, in
\texttt{Semantics}, she would get warned they have not implemented the
\texttt{wellSized} method because class-like interfaces should not contain any
unimplemented methods except for getters.
These errors are captured by automatically re-declaring members with inheritance relationships.
%The \texttt{of} method is  returns an instance of an anonymous class
%that implements the interface with all getters implemented and its instance is returned.
\end{comment}
