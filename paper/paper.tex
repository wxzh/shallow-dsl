\documentclass[10pt,preprint,numbers,nocopyrightspace]{sigplanconf}

% The following \documentclass options may be useful:

% preprint      Remove this option only once the paper is in final form.
% 10pt          To set in 10-point type instead of 9-point.
% 11pt          To set in 11-point type instead of 9-point.
% numbers       To obtain numeric citation style instead of author/year.

\usepackage{amsmath}
\usepackage{color}
\usepackage{listings}
\usepackage{balance}
\usepackage{amsmath}
\usepackage{stmaryrd}
\usepackage{graphicx}
\usepackage{amssymb}
\usepackage{fancyvrb}
\usepackage{url}
\usepackage{pstricks,pst-node,pst-tree}
\usepackage{theorem}
\usepackage{bbm}
\usepackage{pgf}
\usepackage{multirow}
\usepackage{rotating}
\usepackage{verbatim}
\usepackage{graphicx}
\usepackage{wrapfig}
\usepackage{xspace}
\usepackage{mdwlist}
\usepackage{morefloats}
\usepackage[T1]{fontenc}
\usepackage[scaled=0.85]{beramono}
\usepackage{mathptmx}
\usepackage[normalem]{ulem}
\usepackage{booktabs}
\usepackage{tablefootnote}
\usepackage{enumitem}
\usepackage{stmaryrd}
\usepackage{syntax}

\newcommand{\name}{\textbf{@Family}\xspace}
\newcommand{\dsl}{\emph{scans}\xspace}
\newcommand{\interp}{\textsc{Interpreter}\xspace}

%\newcommand{\authornote}[3]{}
\newcommand{\authornote}[3]{{\color{#2} {\sc #1}:#3}}
\newcommand{\weixin}[1]{\authornote{weixin}{cyan}{#1}}
\newcommand{\bruno}[1]{\authornote{bruno}{red}{#1}}
% \setlist[itemize]{noitemsep,nolistsep}
% \setlist[enumerate]{noitemsep,nolistsep}

\lstdefinelanguage{JavaScalaHaskell}{
  morekeywords={public,int,interface,implements,default,
    abstract,case,void,catch,class,def,static,%
    do,else,extends,false,final,finally,%
    for,if,implicit,import,match,mixin,%
    new,null,object,override,package,%
    private,protected,requires,return,sealed,%
    super,this,throw,trait,true,try,%
    type,var,val,while,yield,with,data,type,newtype},
  otherkeywords={=>,<-,<\%,<:,>:,\#,@},
  sensitive=true,
  morecomment=[l]{//},
  morecomment=[n]{/*}{*/},
  morestring=[b]",
  morestring=[b]',
  morestring=[b]"""
}

\lstset{ %
language=JavaScalaHaskell,                % choose the language of the code
columns=fullflexible,
keepspaces=true,
lineskip=-1pt,
basicstyle=\ttfamily\small,       % the size of the fonts that are used for the code
numberstyle=\ttfamily\tiny,      % the size of the fonts that are used for the line-numbers
stepnumber=1,                   % the step between two line-numbers. If it's 1 each line will be numbered
numbers=none,
numbersep=5pt,                  % how far the line-numbers are from the code
backgroundcolor=\color{white},  % choose the background color. You must add \usepackage{color}
showspaces=false,               % show spaces adding particular underscores
showstringspaces=false,         % underline spaces within strings
showtabs=false,                 % show tabs within strings adding particular underscores
%  frame=single,                   % adds a frame around the code
tabsize=2,                  % sets default tabsize to 2 spaces
captionpos=none,                   % sets the caption-position to bottom
breaklines=true,                % sets automatic line breaking
breakatwhitespace=false,        % sets if automatic breaks should only happen at whitespace
title=\lstname,                 % show the filename of files included with \lstinputlisting; also try caption instead of title
escapeinside={(*}{*)},          % if you want to add a comment within your code
keywordstyle=\ttfamily\bfseries,
aboveskip=3pt,
belowskip=-1pt
% commentstyle=\color{Gray},
% stringstyle=\color{Green}
}
\begin{document}

\special{papersize=8.5in,11in}
\setlength{\pdfpageheight}{\paperheight}
\setlength{\pdfpagewidth}{\paperwidth}

\conferenceinfo{CONF 'yy}{Month d--d, 20yy, City, ST, Country}
\copyrightyear{20yy}
\copyrightdata{978-1-nnnn-nnnn-n/yy/mm}
\copyrightdoi{nnnnnnn.nnnnnnn}

% Uncomment the publication rights you want to use.
%\publicationrights{transferred}
%\publicationrights{licensed}     % this is the default
%\publicationrights{author-pays}

\titlebanner{banner above paper title}        % These are ignored unless
\preprintfooter{short description of paper}   % 'preprint' option specified.

\title{Shallow EDSLs and Object-Oriented Programming}

\authorinfo{}
           {}
           {}
%\authorinfo{Name}
%           {Affiliation2/3}
%           {Email2/3}

\maketitle

\begin{abstract}

Shallow Embedded Domain Specific Languages (EDSLs) use
\emph{procedural abstraction} to directly encode a DSL into an existing host language. Procedural abstraction has
been argued to be the essence of Object-Oriented Programming (OOP). 
%Given
%that OO languages have evolved over more than 50 years
%to improve the use of procedural abstraction, they ought to have some
%advantages to encode shallow EDSLs.
This paper argues that OOP abstractions 
(including \emph{inheritance}, \emph{subtyping} and
\emph{type-refinement}) 
increase the modularity and reuse of shallow
EDSLs when compared to classical procedural abstraction. We make this
argument by taking a recent paper by Gibbons and Wu, where procedural
abstraction is used in Haskell to model a simple EDSL, and we recode
that EDSL in Scala. From the \emph{semantic}
and \emph{modularity} point of view the Scala version has clear advantages 
over the Haskell version. 

%To alleviate some of the syntactical disadvantages of Java, we create
%an annotation inspired by \emph{family polymorphism} and a recent
%solution to the Expression Problem.
%The annotation uses transparent code generation techniques to
%automatically eliminate large portions of boilerplate code.
%To further illustrate the applicability of our tool and techniques, we conduct
%several case studies using larger DSLs from the literature.
\end{abstract}

\begin{comment}
\category{CR-number}{subcategory}{third-level}

% general terms are not compulsory anymore,
% you may leave them out
\terms
term1, term2

\keywords
keyword1, keyword2
\end{comment}

%===============================================================================
\section{Introduction}

Since Hudak's seminal paper on Embedded DSLs (EDSLs)~\cite{}, existing
languages (such as Haskell) have been used to directly encode
DSLs. Two common approaches to EDSLs are the so-called \emph{shallow}
and \emph{deep} embeddings. The origin of that terminology can be
attributed to Boulton's work~\cite{}. The difference between these
two styles of embeddings is commonly described as follows:

\begin{quote}
\emph{With a deep embedding, terms in the DSL are implemented simply to
construct an abstract syntax tree (AST), which is subsequently
transformed for optimization and traversed for evaluation. With a
shallow embedding, terms in the DSL are implemented directly by
their semantics, bypassing the intermediate AST and its traversal.}\cite{gibbons15folding}
\end{quote}

%\begin{comment}
%This definition is widely accepted and similar definitions appear in
%many other works~\cite{}. 
%We argue that this definition is vague,
%and often leads to some contradicting claims. 

Although the above definition is quite reasonable and widely accepted,
it leaves some space to (mis)interpretation. For example it is unclear 
how to classify an EDSL using the {\sc Composite} or {\sc Interpreter} 
patterns in OOP. Would this OO approach be
classified as a shallow or deep embedding? We believe arguments can be
made both ways. Since the {\sc Composite}
pattern is normally accepted to be a way to encode ASTs, it would be
reasonable to say that \emph{according to definition of deep embedding
  above, the OO approach classifies as a deep
  embedding}. However, as we shall argure in the remainder of the
paper, another possible interpretation is that the OO approach is
really a shallow embedding. At least some authors~\cite{} seem to agree with 
the latter interpretation, but we believe that the current definition
leaves some space open to interpretation.

\begin{comment}
For example, in their work
on EDSLs~\cite{}, Gibbons and Wu claim that deep embeddings (which
encode ASTs using algebraic datatypes in Haskell) allow adding new DSL
interpretations easily, but they make adding new language constructs
difficult. In contrast Gibbons and Wu claim that shallow embeddings
have dual modularity properties: new cases are easy to add, but new
interpretations are hard.  However what if, instead of using Haskell
and algebraic datatypes, one uses an OO language to encode an AST, for
example with the {\sc Composite} pattern.  Would this OO approach be
classified as a shallow or deep embedding? We believe arguments can be
made both ways. Since the {\sc Composite}
pattern is normally accepted to be a way to encode ASTs, it would be
reasonable to say that \emph{according to definition of deep embedding
  above, the OO approach classifies as a deep
  embedding}. Unfortunatelly this interpretation could be problematic.
As the Expression Problem~\cite{} tell us,
in the OO approach adding new language constructs is easy, but adding
interpretations is hard. Thus this would contradict Gibbons and Wu's
claims, since we have an AST representation (i.e. a deep embedding)
with the modularity properties of shallow embeddings.

We believe that the core of problem is that ASTs can be represented in
multiple ways. In particular, it is well know that functions alone are
enough to encode datastructures such as ASTs (via Church
encodings~\cite{}).  Distinguishing deep and shallow embeddings based
solely on whether a ``real'' datastructure is being used or not is
misleading.  Moreover, it gives the impression that shallow embeddings
are significantly less expressive than deep embeddings, because they
do not have access to the datastructure.
Gibbons and Wu themselves feel uneasy with the definition of shallow 
embeddings when they say:
``\emph{So it turns out that the syntax of the DSL is not really as ephemeral
in a shallow embedding as Boulton's choice of terms suggests.}''
\end{comment}

To avoid confusion and make the terminology more precise, we first 
propose distinguishing EDSLs in terms of the data
abstraction used to model the language constructs instead.  We follow
Reynold's classification~\cite{} of data abstractions:
\emph{procedural abstraction} and \emph{user-defined types}. It is
clear that shallow embeddings use \emph{procedural
  abstraction}~\cite{}: the DSLs are modelled by interpretation
functions. Therefore, the other implementation option for EDSLs is to use
\emph{user-defined types}. In Reynolds terminology user-defined types
mean disjoint union types, which are nowadays commonly available in
modern languages as \emph{algebraic datatypes}. Disjoint union types 
can also be emulated in OOP using the {\sc Visitor} pattern. A
distinction based on data abstraction provides a remedy for possible
misinterpretation. An EDSL implemented with algebraic
datatypes falls into the category of user-defined types (deep embedding), 
while a Composite-based OO implementation falls under procedural
abstraction (shallow embedding). For the rest of the paper we identify shallow EDSLs 
with EDSLs implemented using procedural abstraction.

This paper shows that OOP languages have
advantages for the implementation of shallow embeddings.
As Cook~\cite{} argued procedural abstraction is the essence of
Object-Oriented Programming. If we accept Cook's view, the
implementation of a shallow DSL in OOP languages should simply
correspond to a standard object-oriented program. Given
that Object-Oriented Languages have evolved over more than 50 years 
to improve the use of procedural abstraction, they ought to have some 
advantages to encode shallow EDSLs. 

We show that OOP abstractions, including \emph{inheritance}
and \emph{subtyping}, increase the modularity and reuse of shallow
DSLs when compared to classical procedural abstraction. We make this
argument by taking a recent paper by Gibbons and Wu, where procedural
abstraction is used in Haskell to model a simple EDSL, and we recode
that EDSL in Java. Although from the \emph{syntactical} point of view
there are obvious disadvantages in the Java version, from the semantic
and modularity point of view the Java version has clear advantages.


This paper has two goals:

\begin{itemize}

\item To argue that 

\end{itemize} 
\section{Shallow Object-Oriented Programming}\label{sec:oo}

\begin{comment}
Weixin writes this part.

Argue that shallow embeddings and straightforward OO 
programs are essentially the same thing. 

Start from a simple shallow DSL in Haskell, 
and iterate throught it until you reach a form 
that looks like an OO program.

Show how todo transformations in Shallow embeddings
using the insight of how to do transformations in OO
programs.

Show the correponding Java programs and the Java program 
with transformation that we can port back to Haskell.
\end{comment}

This section shows that an OO approach and shallow embeddings using
procedural abstraction are closely related.  We use a subset of the
DSL presented in Gibbons and Wu's paper~\cite{gibbons2014folding} as
the running example.  We first give the original shallow embedded
implementation in Haskell and rewrite it towards an ``OO style''.
Then translating the program into an OO language becomes straightforward.

\subsection{A DSL for Parallel Prefix Circuits}
Consider a DSL named \dsl that models parallel circuits.
Its BNF grammar is given below:
% BNF grammar should not contain left recursion and left factor, rewrite it
% using descent recursion
% \setlength{\grammarparsep}{20pt plus 1pt minus 1pt} % increase separation between rules
\setlength{\grammarindent}{5em} % increase separation between LHS/RHS

\begin{grammar}
<circuit> ::= `fan' <positive-number>
\alt `id' <positive-number>
\alt `beside' <circuit> <circuit>
\end{grammar}

\noindent The DSL has three constructs: two primitives
\emph{id} and \emph{fan} and one combinator \emph{beside}.
Their meanings are: \emph{id n} contains \emph{n} isolated vertical wires;
\emph{fan n} has \emph{n} vertical wires with its first wire connected to
all the remaining wires from top to bottom; $beside\ c_1\ c_2$ joins two circuits
$c_1$ and $c_2$ horizontally. Simple circuits can be described with these three constructs.
%For example, Figure~\ref{} visualize the circuit \emph{beside (fan 3) (id 3)}.

\subsection{Shallow Embeddings and OOP}\label{subsec:shallow}
Shallow embeddings define a language directly through encoding its semantics
using procedural abstraction. In the case of \dsl,
an shallow embedded implementation should conform to the following
types:

\lstinputlisting[linerange=2-2]{./code/shallowCircuit.hs}%APPLY:CIRCUIT_TYPE
\lstinputlisting[linerange=5-7]{./code/shallowCircuit.hs}%APPLY:TYPES
The type \lstinline{Circuit}, representing the semantic domain, is to be filled in with a concrete type according to the semantics.
Suppose that the semantics of \dsl is to calculate the width of a
circuit. The definitions would be:
\lstinputlisting[linerange=11-14]{./code/shallowCircuit.hs}%APPLY:CIRCUIT1
%As shallow embedding is semantics-oriented,
%\bruno{why is ``id'' in bold?}
%\lstinline{Circuit} is just a type
%synonym of \lstinline{Int} and it is already an integer after circuit construction.
%For example, we will immediately get $6$ for \lstinline{beside (fan 3) (id 3)}.
Note that, for this tiny DSL, the Haskell domain is simply
\lstinline{Int}. This domain is a degenerate case of
procedural abstraction, where \lstinline{Int} can be viewed 
as a no argument function. In Haskell, due to laziness, \lstinline{Int}
is a good representation. In a call-by-value language 
a no-argument function \lstinline{() -> Int} would be more
appropriate to deal correctly with potential control-flow 
language constructs. More realistic shallow DSLs, such as parser 
combinators~\cite{leijen01parsec}, tend to have more complex functional domains.

\begin{comment}

A simple rewriting of the previous program is to wrap the result into an
datatype, getting back the value through pattern matching:
\lstinputlisting[linerange=18-25]{./code/shallowCircuit.hs}%APPLY:CIRCUIT2
% Then circuit construction gives back a \lstinline{Circuit2} which contains an
% integer value representing the width and \lstinline{width2} extracts out the value.
% Is this definition still shallow embedding?
%One may be curious whether this variant implementation is still shallow embedding or not. The answer is yes because ... % why it is shallow

\end{comment}

\paragraph{Towards OOP}
A simple, \emph{semantics preserving}, rewriting of the above program is given
below, where a record with a sole field captures the domain and is declared as a \lstinline{newtype}:
\lstinputlisting[linerange=2-5]{./code/Rewrite.hs}%APPLY:CIRCUIT3
%\bruno{width3 should now be width2}
%We no longer need a separate \lstinline{width3} function as the field name is the extractor.
The implementation is still shallow because \lstinline{newtype} does not add any operational
behaviour to the program, and hence the two programs are effectively the
same.  However, having fields makes the program look more like an 
OOP program.

\paragraph{Porting to Scala}
Indeed, we can easily translate the Haskell program into an OO
language like Scala:
\lstinputlisting[linerange=2-15]{../src/width/Circuit.scala}%APPLY:CIRCUIT_SCALA
The record type maps to the trait \lstinline{Circuit} and field
declaration becomes a method declaration.
Each case in the semantic function corresponds to a trait and its parameters become fields of that trait.
And these traits extend \lstinline{Circuit} and implement \lstinline{width}.


This implementation is essentially how we would model \dsl with an OO language in the first
place, following the \interp pattern (which uses \textsc{Composite} pattern to
organize classes). A minor difference is the use of
traits, instead of classes. Using traits instead of
classes enables some additional modularity via multiple (trait-)inheritance.
In summary, shallow embeddings and straightforward OO programming are closely
related.
%It may worth mentioning that deep embedding is closely related to the \textsc{Visitor} pattern.

\section{Transformations in Shallow Embeddings}
Transformations are typically regarded as the privilege of deep embeddings.
Since there is not a data structure persisting the AST, how to define transformations
becomes unclear.
It is not completely true, especially for the OO approach.
This section illustrates how to define various transformations, the missing
kind of interpretations in Gibbons and Wu's paper, in shallow embeddings.
%In fact, transformations can still be defined in shallow embedding.

\subsection{Desugaring}
Desugaring is one kind of transformations, which eliminates some
language constructs without sacrificing the expressiveness of the whole language.
In \dsl, the \emph{id} construct is such a syntactic sugar that can
be rewritten via the following formula:
$$id\ n = \overbrace{\ beside\ (fan\ 1)}^{\text{repeat }n-1\text{ times}}\ (fan\ 1)$$
which states that \emph{id n} can be represented as \emph{n} of \emph{fan 1} combined with \emph{beside}.
%Using the above formula, we can define a transformation that desugars every
%occurrence of \emph{identity} to a combination of \emph{fan} and \emph{beside}.

Figure~\ref{code:desugar} shows how to eliminate \lstinline{Id} construct in Scala.
\begin{figure}
\lstinputlisting[linerange=4-23]{../src/desugar/Circuit.scala}%APPLY:DESUGAR_SCALA
\caption{OO implementation of desugaring}
\label{code:desugar}
\end{figure}
The newly introduced method \lstinline{desugar} in \lstinline{Circuit} returns a
\lstinline{Circuit} reflecting the nature of a transformation. The implementation of \lstinline{desugar} for each case is
straightforward: \emph{fan n} returns itself; $beside\ c_1\ c_2$ recursively desugars $c_1$ and $c_2$ then wraps back
their results into an instance of \lstinline{Beside}; \emph{id n} just mimics the
formula.

%\weixin{define a show function in the and show the result after desugaring?}
Inspired by the Scala implementation, we can port it back to Haskell:
\lstinputlisting[linerange=6-21]{./code/desugar.hs}%APPLY:DESUGAR
which relies on the laziness of Haskell.

\subsection{Sophisticated Optimizations}
Sophisticated optimizations may need to inspect multiple internal
representations of values.
Suppose we want to merge consecutive vertical wires:
$$
beside\ (id\ m)\ (id\ n) = id\ (m + n)
$$
This rewrite rule applies only when both $c_1$ and $c_2$ of the $beside\ c_1\ c_2$
are $id$, which has to inspect the internal representation of $c_1$ and $c_2$ simultaneously.
However, pattern matching can only be simulated only at the the top level, i.e. $beside$.
There is no easy way to destruct $c_1$ and $c_2$.
Fortunately, Scala has built-in support for pattern matching. For extensibility reasons, we
can not simply decorate classes with \lstinline{case} modifier.
% as it is not possible to have a case class that extends another case class.
Alternatively, we manually define the \lstinline{unapply} method, a.k.a \emph{extractor}, in the companion object. Here comes the implementation
of the above optimization:
\lstinputlisting[linerange=4-22]{../src/optimizations/Circuit.scala}%APPLY:MERGEIDS_SCALA

For other OO languages, it is still possible to simulate
pattern matching through test methods or type test and type casts,
although they may not be as elegant as extractors~\cite{emir2007matching}.
Interested reader can refer to \ref{} for implementations using these approaches.

\begin{comment}
\paragraph{Test methods.}
We can introduce some test methods to the hierarchy for determining the class
type of object and extract information from it:
\begin{lstlisting}
  def fromId: Option[Int] = ...
\end{lstlisting}
Then
\begin{lstlisting}
(for {
  i1 <- c1.fromId; i2 <- c2.fromId
} yield new Id(i1+i2))
.orElse(Some(new Beside(t1,t2))).get
\end{lstlisting}

\paragraph{Type test and type cast.}
Each constructs is of different type. We can test its type and convert.
\begin{lstlisting}
(t1,t2) match {
  case (i1: Id,i2: Id) => new Id(i1.n + i2.n)
  case _ => new Beside(t1,t2)
}
\end{lstlisting}

\paragraph{Extractors.}
Scala supports pattern matching.
one can manually implement an extractor, a.k.a \lstinline{unapply}
method, in the companion object to tell how to destruct an object.
\begin{lstlisting}
object Id {
  def unapply(c: Identity) = Some(c.n)
}

(t1,t2) match {
  case (Id(n1),Id(n2)) => Id(n1+n2)
  case _ => Beside(t1,t2)
}
\end{lstlisting}

Note that we can not simply decorate classes with \lstinline{case} modifier to
automatically generate the \lstinline{unapply} method -
it is not possible to have a case class that extends another case class.
\end{comment}
\section{Modular EDSLs in OOP}
This section shows that procedural abstraction mechanisms provided by OO
languages improves the modularity and reusability of EDSLs, in particular
subtyping and inheritance although most OO languages blend them together in one
syntactic form.

\subsection{Subtyping}
Subtyping allows us separate interfaces and implementations, which
encapsulates details of implementations and yields flexible design of EDSLs.
Also, it forms the foundation of the \textsc{Composite} pattern,
which enables us to composite instances of subtypes where a supertype is needed,
following the Liskov substitution principle.

Figure~\ref{code:base} illustrates the use of subtyping,
where all the classes are subtype of the trait \lstinline{Circuit}.

We can easily introduce new language constructs to \dsl by adding a new class to
the hierarchy.
For example, an \lstinline{above} combinator that combines two circuits vertically:
\begin{lstlisting}
class Above(val c1: Circuit, val c2: Circuit) extends Circuit {
  def width = c1.width
}
\end{lstlisting}
The new class works seamlessly with existing classes, as it is also a subtype of
\lstinline{Circuit}. We can construct a circuit by a combination of old
and new constructs:\\
\lstinline{new Beside(new Above(new Fan(2),new Id(2)),new Id(3))}

% type refinements
% record subtyping.

\subsection{Inheritance}
Inheritance is critical for code reuse, which allows
a class to obtain functionalities from other existing classes
without duplicating the code.
Moreover, the inherited functionality can be refined through method overriding
and new functionalities can be introduced via defining new methods and
fields.

We have already shown how to reuse an EDSL through inheritance in Figure~\ref{code:desugar}.
Instead of copying and pasting \lstinline{width} definition case by case
from the old implementation to the new one, we define the new language through inheriting existing
classes, e.g. \lstinline{class Fan extends base.Fan}.
Additionally, we introduce a new method \lstinline{desugar} to the hierarchy.
and gives a default implementation in \lstinline{Circuit} and override it when needed.
Note that we do covariant type-refinements on fields of type \lstinline{Circuit} so that
we can call \lstinline{desugar} method on them.

% Overriding in FP
One may argue that similar code reuse can be simulated in FP through reusing a record instance.
In fact, it is more like composition than inheritance from the OO perspective.
Inheritance plus dynamic dispatch maximize the code reuse for composition.
Consider a simple example:

\begin{lstlisting}[language=haskell]
data T = T { f :: Int, g :: Int }
a = T { f = g a, g = 0 }
b = a { g = 1 }
\end{lstlisting}
Here, we want to reuse \lstinline{a} but ``override'' its \lstinline{g} field when defining \lstinline{b}.
However, the expected result of calling \lstinline{f b} is 1 but it is indeed 0.
The reason is that the \lstinline{f} field ``inherited'' from \lstinline{a} calls
\lstinline{g} on the old instance rather than the new one.

To make it clear, Figure~\ref{code:compare} give the corresponding OO implementation using composition
and inheritance respectively.
\begin{figure}
\begin{tabular}{lll}
\begin{minipage}{.17\textwidth}
\begin{lstlisting}
trait T {
  def f: Int
  def g: Int
}
class A extends T {
  def f = g
  def g = 0
}
\end{lstlisting}
\end{minipage}
&
\begin{minipage}{.17\textwidth}
\begin{lstlisting}
class B(a: A) extends T {
  def f = a.f
  def g = 1
}
new B(new A).f// 0
\end{lstlisting}
\end{minipage}
&
\begin{minipage}{.17\textwidth}
\begin{lstlisting}
class B extends A {
  override def g = 1
}
(new B).f// 1
\end{lstlisting}
\end{minipage}
\end{tabular}
\caption{Composition (left) vs Inheritance (right)}
\label{code:compare}
\end{figure}
In the case of inheritance, calling \lstinline{f} will execute the overridden \lstinline{g}
as \lstinline{this} points to \lstinline{B} rather than \lstinline{A}.
For compositions, a method can not be reused if when the implementation of what
they depend on changes like \lstinline{f} in \lstinline{A}, even if the logic holds.
OO languages provide these two options for reusing code.

\subsection{Extensibility of EDSLs}
As EDSLs evolve along the time, the need for new syntax and new semantics arises.
This requires host languages equipped with extensibility mechanisms.
OO languages give simple solution to the Expression Problem~\cite{eptrivially16}
and hence are suitable host languages in terms of extensibility.

We have already shown how to extend \dsl on these two dimensions modularly in an
OO language: a new language construct \lstinline{Above} and a new interpretation function \lstinline{desugar}.

Gibbons and Wu introduce new interpretations in a way similar to
Figure~\ref{code:haskell} by appending the definition to a tuple,
which requires revision on the original code.
Though they also present a modular solution based on~\ref{swierstra2008data},
the complexity of encoding increases significantly. Moreover, how to dependent
interpretation modularly is not clear.

\subsection{Discussion}
There are certain kind of methods that are not reusable using the OO approach we
presented, including transformations (or producer methods) and binary
methods as discussed in~\ref{eptrivially16}.
For transformations, we have to refine their return type in extensions and hence
the original code can not be reused, resulting code duplications.
For binary methods like equality, there is no way to refine the argument type.
It is still possible to make it reusable by using virtual types in Scala to capture the type of \lstinline{this}~\ref{zenger}.
However, this may significantly complicate the encoding.

\section{Related Work}

\paragraph{Deep and Shallow Embeddings}
Shallow embedding yields flexible and concise EDSLs while deep embedding makes
it easy to define optimizations.
There are a lot of work~\cite{svenningsson2012combining,
  Jovanovic:2014:YCD:2658761.2658771, scherr2014implicit} trying to blend these two
approaches to enjoy benefits from both.
They typically encode the surface language with a shallow embedding and
then generate or translate to a deep embedded version for allowing optimizations.
%Hofer and Ostermann~\cite{hofer2010modular} propose to provide both embedding through implementing internal and external visitor at the same time so that clients can choose for a particular interpretation;
The OO approach we present retains the simplicity of shallow embedding while
makes it possible to implement optimizations.

\paragraph{Modularity of EDSLs}
Languages that solves the EP are capable of being host languages for
implementing modular EDSLs. But many of these solutions require sophisticated
type parameterization, advanced features or heavy encoding which impose
complexity to the DSL implementers or even clients, preventing practical use.
For example, solutions based on the \textsc{Visitor}
pattern~\cite{oliveira09modular,hofer2010modular} %more references
Relatively simple solution like Object Algebras~\cite{bruno12oa} may not support
optimizations well.
Our approach, however, is simple and supports transformations.

%\section{Overview}
\weixin{title: Evolution on DSLs?}

\begin{comment}
Weixin writes this one.

Go over jeremy's examples, maybe having only 3 diagram 
constructs instead of 5 for space reasons.

Use five constructs to show extensibility.

Think of how to introduce our tool? Using the Jeremy's examples? 
or introducing before with some other examples?

\end{comment}

%As language evolves, the need of new syntax often arises.
As DSLs evolve along the time, the demand for new syntax and and semantics may arise.
It would be good if we can introduce these new features to the DSL \emph{modularly}.
%This is actually a hard problem, known as the Expression Problem (EP)~\cite{wadler}, which
This, however, requires the host language equipped with two dimensions of
extensibility.
In this section, we argue that from extensibility perspective, OO languages are
better candidates as host languages than FP languages.
To demonstrate this, we try to extend \dsl with new syntax and new semantics in
the setting of shallow embedding. Adding new syntax is both easy for both FP
and OOP. Adding new semantics, however, is hard in FP. Although it is possible
in OOP, the solution requires some boilerplate code. We hence developed \name
for defining modular extensions easily.

\subsection{Initial System}
Before introducing any extensions, we need to rewrite the initial implementation for \dsl shown in Figure~\ref{}:

\lstinputlisting[linerange=11-26]{../src/paper/Circuit.java}%APPLY:INIT
We change the implementation language constructs from classes to interfaces and
their fields to unimplemented getter methods. The purpose of this modification
is to allow types of the fields to be refined in future, which is vital for
retaining extensibility.

\subsection{Adding New Syntax}
Shallow embedding makes it easy to add new syntax, both in FP and OOP.
Suppose that we would like to add two new constructs to the language:
the first is \emph{stretch ns c}, which inserts additional wires into the circuit \emph{c} by summing up \emph{ns};
the second is $above c_1 c_2$, which combines two circuits of the same width vertically.
With these new constructs, more complex circuit can be constructed.
For example, Figure~\ref{} shows the circuit constructed by \emph{stretch [2,2,2] (fan 3) `beside` fan 1}.

To accomplish this goal in Haskell, we can simply define two more cases,
\texttt{stretch} and \texttt{above}, for the semantic function:
\lstinputlisting[linerange=59-60]{./code/shallowCircuit.hs}%APPLY:SYNTAX_HS
Similarly, defining two new interfaces
\texttt{Stretch} and \texttt{Above} that both extend \texttt{Circuit} and
implement the \texttt{width} method is all we need to do in Java:

\lstinputlisting[linerange=29-40]{../src/paper/Circuit.java}%APPLY:SYNTAX

\subsection{Adding New Semantics}
The new combinators, however, can not apply to arbitrary circuits - they have
some invariants in their definitions.
To make sure that a circuit is constructed correctly, we need to expand the
semantics of the language for doing such checks.

Adding new semantics, however, becomes hard for shallow embedding.
Gibbons and Wu worked around this problem in the following way:
\lstinputlisting[linerange=47-51]{../src/paper/Circuit.java}%APPLY:HS_SEMANTICS

which is similar to how we add \texttt{desugar} to the implementation in Section~\ref{}.
The difference is that they use a tuple instead of a record to merge the
semantic domains, then define the two semantic functions simultaneously for each
case, and split the definitions through projections on the tuple in the end.
This solution, however, modifies existing code, breaking the requirement of EP.

Conversely, the support of covariant type-refinements and inheritance for OO
languages allows us to add new semantics in a \emph{modular} way:
%Different from records or tuples, interfaces are extensible.
\lstinputlisting[linerange=47-75]{../src/paper/Circuit.java}%APPLY:SEMANTICS

Interface \texttt{CircuitWS} extends the original interface and declare a new
semantic function \texttt{wellSized} inside.
Then all existing cases should extend both their corresponding original
implementation and \texttt{CircuitWS} and implement the new method
\texttt{wellSized}. Also, all the occurrences of \texttt{Circuit} are
refined as \texttt{CircuitWS} so that we can call \texttt{wellSized} on inner circuits returned by getters.
As Java does not support type-refinements on fields, we hence implement these
constructs as interfaces rather than classes.

\subsection{\name's Support}
There exists some boilerplate for the Java solution presented above:
\begin{itemize}
  \item Interfaces that represent language constructs should be instantiated for
    creating objects;
  \item Type-refinements should be done manually
  \item The inheritance relation needs to be repeatedly stated for all interfaces.
\end{itemize}
Also, programmers would not get warned if they forget to extend all the
constructs with new semantics.

\name addresses all these problems through embracing family polymorphism and code instrument.
By using \name, we can refactor the extensions in the way shown in Figure\ref{}.
\lstinputlisting[linerange=80-83]{../src/paper/Circuit.java}%APPLY:FAMILY
\lstinputlisting[linerange=112-122]{../src/paper/Circuit.java}%APPLY:FAMILY_SYNTAX
\lstinputlisting[linerange=126-141]{../src/paper/Circuit.java}%APPLY:FAMILY_SEMANTICS

Note that \name generates \texttt{of} method for class-like interfaces for
constructing objects. The static method \texttt{of} returns an instance of an anonymous class
that implements the interface with all
getters implemented and its instance is returned.
Moreover, \name has a good support family polymorphism.
Client users need only explicitly declared the dependencies on other families
through (...). \name will re-declare all members from the base families in the extended family.
And the dependencies between members of families are automatically deducted and
implicitly expressed in member re-declaration.

% Multiple interpretation
% Dependent interpretation
% Context-sensitive interpretation

%\section{Code Generation}

%Haoyuan should write this part.
%How to generate code for family polymorphism!

In this section, we present an overview of the code that \name generates. The syntax and type system are
consistent with the Java language. We use translation functions to illustrate our code generation.
To make it clearer, we split the process of code generation into two parts, in which case we introduce a
new annotation \lstinline{@FamilyOf}, which is not defined in the source code but only helps to explain
the translation. In our implementation, the annotation processing is a combination of \name
and \lstinline{@FamilyOf}. Below are the translation functions:

\begin{lstlisting}
  (*$\llbracket$*)@Obj interface (*$I_0$*) extends (*$\overline{I_s}$*) {(*$\overline{meth}$*) (*$\overline{I}$*)}(*$\rrbracket$*)
= @ObjOf interface (*$I_0$*) extends (*$\overline{I_s}$*) {(*$\overline{meth'}$*) (*$\overline{I'}$*)}
\end{lstlisting}

\subsection{@Obj}
\lstinline{@Obj} builds the structure of family polymorphism for the annotated type. More specifically, \lstinline{@Obj} tackles two
tasks: (1) building the dependencies (subtyping relations) between new family members and old ones; (2) refining field types. To this
aim, \lstinline{@Obj} re-declares all member types and field methods from the base families. The detection runs recursively throughout
the nested interfaces in base families. For member types with same names, the re-declaration in the annotated type is generated or modified
with supertypes filled in for dependencies.

\subsection{@ObjOf}
The second part of annotation processing is abstracted here using the newly introduced annotation
\lstinline{@ObjOf}. It recursively generates constructer methods \lstinline{of} for the annotated interface
and all nested interfaces. Furthermore, \lstinline{@ObjOf} supports getters, void and fluent setters, withers and
the general \lstinline{with} methods. That is to say, \lstinline{@ObjOf} generates a static \lstinline{of} method
for a type, returning an instance of that type. Such an instance is implemented using an anonymous class with auto-generated
implementations for all the methods it supports (stated above). On the other hand, users are expected to put underscores as the
prefix of field methods, and consequently \lstinline{of} identifies these field methods and takes them as its arguments.

\begin{figure*}
\begin{lstlisting}
  (*$\llbracket$*)@Family interface (*$I_m$*) extends (*$I_1,\cdots,I_n$*) {(*$\overline{meth}$*) (*$\overline{I}$*)}(*$\rrbracket$*) = (*$\llbracket$*)@Obj interface (*$I_m$*) extends (*$I_1,\cdots,I_n$*) {(*$\overline{meth'}$*) (*$\overline{I'}$*)}(*$\rrbracket$*)
\end{lstlisting}
\hspace{.3in}where $\overline{meth'}=\overline{meth}\ \cup\ $\textsf{fieldMethods(}$I_1$\textsf{)}$\ \cup\cdots\cup\ $\textsf{fieldMethods(}$I_n$\textsf{)}, $\overline{I'}$ = \textsf{newChilds(}$I_1,\cdots,I_n,\overline{I}$\textsf{)}
~\\~\\
(1) $\overline{I'}$ = \textsf{newChilds(}$I_1,\cdots,I_n,\overline{I}$\textsf{)} \textcolor{red}{(definition of newChilds)}
    \begin{itemize}
    \item $\forall I_0\in\overline{I}$, if $\exists I'_0\in\ $\textsf{collectMembers(}$I_1,\cdots,I_n$\textsf{)} and \textsf{name(}$I_0$\textsf{)} = \textsf{name(}$I'_0$\textsf{)}, then \textsf{combine(}$I_0,I'_0$\textsf{)}$\ \in\overline{I'}$
    \item Otherwise, $I_0\in\overline{I'}$
    \end{itemize}
(2) $\llbracket$\lstinline{@Family}\textsf{ interface }$I$\textsf{ extends }$\overline{I_s},\ \overline{I_t}$\textsf{ \{\}}$\rrbracket\ \in$ \textsf{collectMembers(}$I_1,\cdots,I_n$\textsf{)} \textcolor{red}{(definition of collectMembers)}
    \begin{itemize}
    \item $\forall i,$ s.t. $I\in$\textsf{ childs(}$I_i$\textsf{)},
        \begin{itemize}
        \item $\overline{I_s}$ = \textsf{suptypes(}$I_i.I$\textsf{)}
        \item $I_i.I\in\overline{I_t}$
        \end{itemize}
    \end{itemize}
(3) $\overline{I_0}$ = \textsf{childs(}$I$\textsf{)}, $\overline{I}$ = \textsf{suptypes(}$I$\textsf{)} \textcolor{red}{(definition of childs and suptypes)}
    \begin{itemize}
    \item \textsf{ibody(}$I$\textsf{)} = \textsf{interface }$I$\textsf{ extends }$\overline{I}$\textsf{ \{}$\overline{meth}\ \overline{I_0}$\textsf{\}}
    \item \textcolor{red}{need to use ibody(I) to find the declaration. formalize?}
    \end{itemize}
(4) $meth\in\ $\textsf{fieldMethods(}$I$\textsf{)} \textcolor{red}{(definition of fieldMethods)}
    \begin{itemize}
    \item $meth\in\ $\textsf{childs(}$I$\textsf{)}
    \item $meth$ = $I_0$\textsf{ m();}
    \end{itemize}
(5) \textsf{interface }$I$\textsf{ extends }$\overline{I_s}$\textsf{ \{}$\overline{meth}\ \overline{I}$\textsf{\}} = \textsf{combine(}$I_m,I_n$\textsf{)} \textcolor{red}{(definition of combine)}
    \begin{itemize}
    \item \textsf{ibody(}$I_m$\textsf{)} = \textsf{interface }$I$\textsf{ extends }$\overline{I_{s1}}$\textsf{ \{}$\overline{meth_1}\ \overline{I_1}$\textsf{\}}
    \item \textsf{ibody(}$I_n$\textsf{)} = \textsf{interface }$I$\textsf{ extends }$\overline{I_{s2}}$\textsf{ \{}$\overline{meth_2}\ \overline{I_2}$\textsf{\}}
    \item $\overline{I_s}$ = $\overline{I_{s1}}\ \cup\ \overline{I_{s2}}$
    \item $\overline{meth}$ = $\overline{meth_1}\ \cup\ \overline{meth_2}$
    \item If $\exists I_1\in\overline{I_1}, I_2\in\overline{I_2}$, \textsf{name(}$I_1$\textsf{)} = \textsf{name(}$I_2$\textsf{)}, then \textsf{combine(}$I_1,I_2$\textsf{)}$\ \in\overline{I}$
    \item Otherwise $(I\in I_1\ \Delta\ I_2)$, $I\in\overline{I}$.
    \end{itemize}
\caption{Translation of \lstinline{@Family}.}
\end{figure*}

\begin{figure*}
\begin{lstlisting}
  (*$\llbracket$*)@Obj interface (*$I_0$*) extends (*$\overline{I_s}$*) {(*$\overline{meth}$*) (*$\overline{I}$*)}(*$\rrbracket$*) = (*$\llbracket$*)@ObjOf interface (*$I_0$*) extends (*$\overline{I_s}$*) {(*$\overline{meth}$*) (*$\overline{meth'}$*) (*$\overline{\llbracket\textsf{@Obj}\ I\rrbracket}$*)}(*$\rrbracket$*)
\end{lstlisting}
\hspace{.3in}where $\overline{meth'}$ = \textsf{refine(}$I_0,\overline{meth}$\textsf{)}
\begin{lstlisting}
  (*$\llbracket$*)@ObjOf interface (*$I_0$*) extends (*$\overline{I_s}$*) {(*$\overline{meth}$*) (*$\overline{I}$*)}(*$\rrbracket$*) = interface (*$I_0$*) extends (*$\overline{I_s}$*) {(*$\overline{meth}$*) ofMethod((*$I_0$*)) (*$\overline{I}$*)}
\end{lstlisting}
\hspace{.3in}with \textsf{valid(}$I_0$\textsf{)}, \textsf{of} $\notin$ \textsf{dom(}$I_0$\textsf{)}
~\\~\\
(1) $I_0$ \textsf{with}$\#m$\textsf{(}$I\ \_$\textsf{val);} $\in$ \textsf{refine(}$I_0,\overline{meth}$\textsf{)} \textcolor{red}{(part I definition of newChilds) fields with underscore, field and isField}
    \begin{itemize}
    \item \textsf{isWith(mbody(with}$\#m,I_0$\textsf{)}$,I_0$\textsf{)}
    \item \textsf{with}$\#m$ $\notin$ \textsf{dom(}$\overline{meth}$\textsf{)}
    \end{itemize}
(2) $I_0\ \_m$\textsf{(}$I\ \_$\textsf{val);} $\in$ \textsf{refine(}$I_0,\overline{meth}$\textsf{)} \textcolor{red}{(part II definition of newChilds)}
    \begin{itemize}
    \item \textsf{isSetter(mbody(}$\_m,I_0$\textsf{)}$,I_0$\textsf{)}
    \item $\_m$ $\notin$ \textsf{dom(}$\overline{meth}$\textsf{)}
    \end{itemize}
(3) \textsf{valid(}$I_0$\textsf{)} if $\forall m\ \in\ $\textsf{dom(}$I_0$\textsf{)}, let $meth$ = \textsf{mbody(}$m,I_0$\textsf{)}, one of the following cases is satisfied:  \textcolor{red}{(definition of valid)}
    \begin{itemize}
    \item \textsf{isField(}$meth$\textsf{)}, where \textsf{isField(}$I\ m$\textsf{();)} = not \textsf{special(}$m$\textsf{)}
    \item \textsf{isWith(}$meth,I_0$\textsf{)}, where \textsf{isWith(}$I'$ \textsf{with}$\#m$\textsf{(}$I\ x$\textsf{);}$,I_0$\textsf{)} = $I_0 :< I'$, \textsf{mbody(}$m,I_0$\textsf{)} = $I\ m$\textsf{();} and not \textsf{special(}$m$\textsf{)}
    \item \textsf{isSetter(}$meth,I_0$\textsf{)}, where \textsf{isSetter(}$I'$ $\_m$\textsf{(}$I\ x$\textsf{);}$,I_0$\textsf{)} = $I_0 :< I'$, \textsf{mbody(}$m,I_0$\textsf{)} = $I\ m$\textsf{();} and not \textsf{special(}$m$\textsf{)}
    \end{itemize}
(4) \textsf{ofMethod(}$I_0$\textsf{)} = \textsf{static }$I_0$\textsf{ of(}$I_1\ \_m_1,\cdots,I_n\ \_m_n$\textsf{) \{} \textcolor{red}{(definition of ofMethod)}
    \\ \textsf{return new }$I_0$\textsf{() \{}
    \\ $I_1\ m_1$ = $\_m_1$\textsf{;}$\cdots I_n\ m_n$ = $\_m_n$\textsf{;}
    \\ $I_1\ m_1$\textsf{() \{return }$m_1$\textsf{;\}}$\cdots I_n\ m_n$\textsf{() \{return }$m_n$\textsf{;\}}
    \\ \textsf{withMethod(}$I_1,m_1,I_0,\overline{e}_1$\textsf{)}$\cdots$\textsf{withMethod(}$I_n,m_n,I_0,\overline{e}_n$\textsf{)}
    \\ \textsf{setterMethod(}$I_1,m_1,I_0$\textsf{)}$\cdots$\textsf{setterMethod(}$I_n,m_n,I_0$\textsf{)}
    \\ \textsf{\};\}}
    \begin{itemize}
    \item $I_1\ m_1$\textsf{();}$\cdots I_n\ m_n$\textsf{();} = \textsf{fields(}$I_0$\textsf{)}
    \item $\overline{e}_i$ = $m_1,\cdots,m_{i-1},\_$\textsf{val}$,m_{i+1},\cdots,m_n$
    \end{itemize}
(5) $meth$ $\in$ \textsf{fields(}$I_0$\textsf{)} \textcolor{red}{(definition of fields)}
    \begin{itemize}
    \item \textsf{isField(}$meth$\textsf{)}
    \item $meth$ = \textsf{mbody(}$m^{meth},I_0$\textsf{)}
    \end{itemize}
(6) $I_0$ \textsf{with}$\#m$\textsf{(}$I\ \_$\textsf{val) \{return }$I_0$\textsf{.of(}$\overline{e}$\textsf{);\}} = \textsf{withMethod(}$I,m,I_0,\overline{e}$\textsf{)} \textcolor{red}{(definition of withMethod)}
    \begin{itemize}
    \item \textsf{mbody(with}$\#m,I_0$\textsf{)} has the form \textsf{mh;}
    \end{itemize}
(7) $I_0$ $\_m$\textsf{(}$I\ \_$\textsf{val) \{}$m$ = $\_$\textsf{val;return this;\}} = \textsf{setterMethod(}$I,m,I_0$\textsf{)} \textcolor{red}{(definition of setterMethod)}
    \begin{itemize}
    \item \textsf{mbody(}$\_m,I_0$\textsf{)} has the form \textsf{mh;}
    \end{itemize}
\caption{Translation of \lstinline{@Obj} and \lstinline{@ObjOf}.}
\end{figure*}

%\section{Case Study}

Weixin and Haoyuan.

Talk about the 3 case studies

Talk about Pretty Printer case study last and discuss issues with 
extensible transformations.

%\section{Related Work}

\paragraph{Deep and Shallow Embeddings}

\paragraph{Modularity of DSLs?}
Data types a la Carte; Finally Tagless; 
Object Algebras

\paragraph{Family Polymorphism}

\paragraph{ThisType}

\paragraph{Multiple Inheritance}

%\section{Discussion and Limitations}

- Transformations and Binary methods (but argue that 
these types of operations are not used in shallow DSLs 
any way!).

- Java syntax not great for DSLs. 
operator overloding; infix operators (but this is a Java limitation; 
not a limitation of our approach).

- Limitations of the tool: separate compiltions and ... 
generics not fully implemented.

- How to apply our approach to other OO language 
  - C\# has annotations, but can you do AST rewriting?
  - Scala has macros;
  - Other languages? 

%\acks


%Acknowledgments, if needed.

% We recommend abbrvnat bibliography style.

\bibliographystyle{abbrvnat}
\bibliography{paper}

% The bibliography should be embedded for final submission.

\end{document}
