\documentclass[10pt,preprint,numbers,nocopyrightspace]{sigplanconf}

% The following \documentclass options may be useful:

% preprint      Remove this option only once the paper is in final form.
% 10pt          To set in 10-point type instead of 9-point.
% 11pt          To set in 11-point type instead of 9-point.
% numbers       To obtain numeric citation style instead of author/year.

\usepackage{amsmath}
\usepackage{color}
\usepackage{listings}
\usepackage{balance}
\usepackage{amsmath}
\usepackage{stmaryrd}
\usepackage{graphicx}
\usepackage{amssymb}
\usepackage{fancyvrb}
\usepackage{url}
\usepackage{pstricks,pst-node,pst-tree}
\usepackage{theorem}
\usepackage{bbm}
\usepackage{pgf}
\usepackage{multirow}
\usepackage{rotating}
\usepackage{verbatim}
\usepackage{graphicx}
\usepackage{wrapfig}
\usepackage{xspace}
\usepackage{mdwlist}
\usepackage{morefloats}
\usepackage[T1]{fontenc}
\usepackage[scaled=0.85]{beramono}
\usepackage{mathptmx}
\usepackage[normalem]{ulem}
\usepackage{booktabs}
\usepackage{tablefootnote}
\usepackage{enumitem}
\usepackage{stmaryrd}
\usepackage{syntax}

\newcommand{\name}{\textbf{@Family}\xspace}
\newcommand{\dsl}{\emph{scans}\xspace}
\newcommand{\interp}{\textsc{Interpreter}\xspace}

%\newcommand{\authornote}[3]{}
\newcommand{\authornote}[3]{{\color{#2} {\sc #1}:#3}}
\newcommand{\weixin}[1]{\authornote{weixin}{cyan}{#1}}
% \setlist[itemize]{noitemsep,nolistsep}
% \setlist[enumerate]{noitemsep,nolistsep}

\lstdefinelanguage{JavaScala}{
  morekeywords={public,int,interface,implements,default,
    abstract,case,void,catch,class,def,static,%
    do,else,extends,false,final,finally,%
    for,if,implicit,import,match,mixin,%
    new,null,object,override,package,%
    private,protected,requires,return,sealed,%
    super,this,throw,trait,true,try,%
    type,var,val,while,yield,with},
  otherkeywords={=>,<-,<\%,<:,>:,\#,@},
  sensitive=true,
  morecomment=[l]{//},
  morecomment=[n]{/*}{*/},
  morestring=[b]",
  morestring=[b]',
  morestring=[b]"""
}

\lstset{ %
language=JavaScala,                % choose the language of the code
columns=fullflexible,
keepspaces=true,
lineskip=-1pt,
basicstyle=\ttfamily\small,       % the size of the fonts that are used for the code
numberstyle=\ttfamily\tiny,      % the size of the fonts that are used for the line-numbers
stepnumber=1,                   % the step between two line-numbers. If it's 1 each line will be numbered
numbers=none,
numbersep=5pt,                  % how far the line-numbers are from the code
backgroundcolor=\color{white},  % choose the background color. You must add \usepackage{color}
showspaces=false,               % show spaces adding particular underscores
showstringspaces=false,         % underline spaces within strings
showtabs=false,                 % show tabs within strings adding particular underscores
%  frame=single,                   % adds a frame around the code
tabsize=2,                  % sets default tabsize to 2 spaces
captionpos=none,                   % sets the caption-position to bottom
breaklines=true,                % sets automatic line breaking
breakatwhitespace=false,        % sets if automatic breaks should only happen at whitespace
title=\lstname,                 % show the filename of files included with \lstinputlisting; also try caption instead of title
escapeinside={(*}{*)},          % if you want to add a comment within your code
keywordstyle=\ttfamily\bfseries,
aboveskip=3pt,
belowskip=-1pt
% commentstyle=\color{Gray},
% stringstyle=\color{Green}
}
\begin{document}

\special{papersize=8.5in,11in}
\setlength{\pdfpageheight}{\paperheight}
\setlength{\pdfpagewidth}{\paperwidth}

\conferenceinfo{CONF 'yy}{Month d--d, 20yy, City, ST, Country}
\copyrightyear{20yy}
\copyrightdata{978-1-nnnn-nnnn-n/yy/mm}
\copyrightdoi{nnnnnnn.nnnnnnn}

% Uncomment the publication rights you want to use.
%\publicationrights{transferred}
%\publicationrights{licensed}     % this is the default
%\publicationrights{author-pays}

\titlebanner{banner above paper title}        % These are ignored unless
\preprintfooter{short description of paper}   % 'preprint' option specified.

\title{Shallow EDSLs and Object-Oriented Programming}

\authorinfo{}
           {}
           {}
%\authorinfo{Name}
%           {Affiliation2/3}
%           {Email2/3}

\maketitle

\begin{abstract}

Shallow Embedded Domain Specific Languages (EDSLs) use
\emph{procedural abstraction} to directly encode a DSL into an existing host language. Procedural abstraction has
been argued to be the essence of Object-Oriented Programming (OOP). Given
that OO languages have evolved over more than 50 years
to improve the use of procedural abstraction, they ought to have some
advantages to encode shallow EDSLs.

This paper argues that OOP abstractions, including \emph{inheritance}
and \emph{subtyping}, increase the modularity and reuse of shallow
EDSLs when compared to classical procedural abstraction. We make this
argument by taking a recent paper by Gibbons and Wu, where procedural
abstraction is used in Haskell to model a simple EDSL, and we recode
that EDSL in Scala. From the \emph{semantic}
and \emph{modularity} point of view the Scala version has clear advantages 
over the Haskell version. 

%To alleviate some of the syntactical disadvantages of Java, we create
%an annotation inspired by \emph{family polymorphism} and a recent
%solution to the Expression Problem.
%The annotation uses transparent code generation techniques to
%automatically eliminate large portions of boilerplate code.
%To further illustrate the applicability of our tool and techniques, we conduct
%several case studies using larger DSLs from the literature.
\end{abstract}

\begin{comment}
\category{CR-number}{subcategory}{third-level}

% general terms are not compulsory anymore,
% you may leave them out
\terms
term1, term2

\keywords
keyword1, keyword2
\end{comment}

%===============================================================================
\section{Introduction}

Since Hudak's seminal paper on \weixin{EDSLs}~\cite{hudak1998modular}, existing
languages (such as Haskell) have been used to directly encode
DSLs. Two common approaches to EDSLs are the so-called \emph{shallow}
and \emph{deep} embeddings. The origin of that terminology can be
attributed to Boulton et al.'s work~\cite{Boulton92dsl}. The difference between these
two styles of embeddings is commonly described as follows:

\begin{quote}
\emph{With a deep embedding, terms in the DSL are implemented simply to
construct an abstract syntax tree (AST), which is subsequently
transformed for optimization and traversed for evaluation. With a
shallow embedding, terms in the DSL are implemented directly by
their semantics, bypassing the intermediate AST and its traversal.}\cite{gibbons2014folding}
\end{quote}

%\begin{comment}
%This definition is widely accepted and similar definitions appear in
%many other works~\cite{}. 
%We argue that this definition is vague,
%and often leads to some contradicting claims. 

Although the above definition is quite reasonable and widely accepted,
it leaves some space to (mis)interpretation. For example it is unclear 
how to classify an EDSL implemented using the {\sc Composite} or {\sc Interpreter} 
patterns in OOP. Would this OO approach be
classified as a shallow or deep embedding? We believe arguments can be
made both ways. The {\sc Composite} or {\sc Interpreter}
patterns are normally accepted to provide a way to encode ASTs. Thus, 
one possible interpretation is that \emph{according to definition of deep embedding
  above, the OO approach classifies as a deep
  embedding}. 

\begin{comment}
For example, in their work
on EDSLs~\cite{}, Gibbons and Wu claim that deep embeddings (which
encode ASTs using algebraic datatypes in Haskell) allow adding new DSL
interpretations easily, but they make adding new language constructs
difficult. In contrast Gibbons and Wu claim that shallow embeddings
have dual modularity properties: new cases are easy to add, but new
interpretations are hard.  However what if, instead of using Haskell
and algebraic datatypes, one uses an OO language to encode an AST, for
example with the {\sc Composite} pattern.  Would this OO approach be
classified as a shallow or deep embedding? We believe arguments can be
made both ways. Since the {\sc Composite}
pattern is normally accepted to be a way to encode ASTs, it would be
reasonable to say that \emph{according to definition of deep embedding
  above, the OO approach classifies as a deep
  embedding}. Unfortunatelly this interpretation could be problematic.
As the Expression Problem~\cite{} tell us,
in the OO approach adding new language constructs is easy, but adding
interpretations is hard. Thus this would contradict Gibbons and Wu's
claims, since we have an AST representation (i.e. a deep embedding)
with the modularity properties of shallow embeddings.

We believe that the core of problem is that ASTs can be represented in
multiple ways. In particular, it is well know that functions alone are
enough to encode datastructures such as ASTs (via Church
encodings~\cite{}).  Distinguishing deep and shallow embeddings based
solely on whether a ``real'' datastructure is being used or not is
misleading.  Moreover, it gives the impression that shallow embeddings
are significantly less expressive than deep embeddings, because they
do not have access to the datastructure.
Gibbons and Wu themselves feel uneasy with the definition of shallow 
embeddings when they say:
``\emph{So it turns out that the syntax of the DSL is not really as ephemeral
in a shallow embedding as Boulton's choice of terms suggests.}''
\end{comment}

In this paper, to avoid ambiguity, we prefer a more precise meaning
for shallow embeddings as EDSLs implemented using \emph{procedural abstraction}~\cite{reynolds94proceduralabstraction}. Such
interpretation arises from the domain of shallow DSLs being
typically a function, and procedural abstraction being a way to encode
data abstractions using functions. As Cook~\cite{cook09abstraction} argued,
procedural abstraction is also the essence of Object-Oriented
Programming. Thus, according to our definition, the implementation of a shallow
EDSL in OOP languages should simply correspond to a standard
object-oriented program.

\begin{comment}
If we accept Cook's view on OOP, 
a natural way to distinguish implementations of EDSLs is 
in terms of the data abstraction used to model the language
constructs instead. As Reynold's~\cite{reynolds94proceduralabstraction} suggested there are two
types of data abstraction: procedural abstraction and \emph{user-defined
  types}. It is clear that shallow embeddings use \emph{procedural
  abstraction}: the DSLs are modelled by interpretation
functions. Thus, the other implementation option for EDSLs is to
use \emph{user-defined types}. In Reynolds terminology user-defined
types mean disjoint union types, which are nowadays commonly available
in modern languages as \emph{algebraic datatypes}. Disjoint union
types can also be emulated in OOP using the {\sc Visitor} pattern. 

A distinction based on data abstraction is more precise and provides a
remedy for possible misinterpretation. An EDSL implemented with
algebraic datatypes falls into the category of user-defined types
(deep embedding), while a {\sc Composite}-based OO implementation falls
under procedural abstraction (shallow embedding). 
\end{comment}

%At least some authors~\cite{} seem to implicitly
%agree with the latter interpretation.

The main goal of this paper is to argue that OOP languages have
advantages for the implementation of shallow embeddings. 
%This should
%not came as a surprise, as OOP languages have evolved over more than 50
%years to improve the use of procedural abstraction. 
In order to understand how OOP can help with shallow embeddings, lets
first review some of the limitations commonly
found in the literature:

\begin{enumerate}

\item {\bf Single Interpretation} An often stated limitation of
  shallow embeddings is that they only support a single
  interpretation.

\item {\bf No Transformations} Another commonly stated limitation 
of shallow embeddings is that they do not support transformations,
preventing optimizations and other useful transformations.

\end{enumerate}

\noindent Both limitations are often used as motivators to switch to deep embeddings.

We show that OOP abstractions, including \emph{(multiple)
  inheritance}, \emph{subtyping} and \emph{type-refinement}, are
helpful to address those problems. For the first problem, we can
employ a recently proposed design pattern~\cite{eptrivially16}, which provides a simple
solution to the \emph{Expression Problem}~\cite{expPb} in OOP languages. Thus
using just standard OOP mechanims enables \emph{multiple modular
  interpretations} to co-exist and be combined in shallow embeddings.
For the second problem, we show that transformations can be encoded 
with recursive objects. Interestingly enough, for transformations it
is possible to port back the OO solution to
Haskell.

We make our arguments by taking a recent paper by Gibbons and Wu~\cite{gibbons2014folding},
where procedural abstraction is used in Haskell to model a simple
EDSL, and we recode that EDSL in Scala. Although from the
\emph{syntactical} point of view there are minor inconveniences in
the Scala version, from the \emph{semantic} and \emph{modularity} point of view the
Scala version has clear advantages.

\begin{comment}
In summary, our contributions are:

\begin{itemize}

\item {}

\item {\bf Multiple Modular Interpretations for Shallow Embeddings:} 
  We show that with standard OOP mechanisms it is easy to support multiple modular
  interpretations for shallow embeddings.

\item {\bf Transformations for Shallow Embeddings:} We show that
  transformations are encodable with recursive objects. Moreover, this technique
  can be ported back into functional programming as well.

\end{itemize}
\end{comment}

\section{Shallow Object-Oriented Programming}

Argue that shallow embeddings and straightforward OO 
programs are essentially the same thing. 

Start from a simple shallow DSL in Haskell, 
and iterate throught it until you reach a form 
that looks like an OO program.

Show how todo transformations in Shallow embeddings
using the insight of how to do transformations in OO
programs.

Show the correponding Java programs and the Java program 
with transformation that we can port back to Haskell.

\begin{verbatim}
{-
type Exp = Int

lit x = x

add e1 e2 = e1 + e2
-}

{-
newtype Exp = E Int

lit x = E x

add (E e1) (E e2) = E (e1 + e2)
-}

data Exp = E {
  eval :: Int,
  transform :: Exp
}

lit x = E {eval = x, transform = lit x}

add e1 e2 = E {
  eval = eval e1 + eval e2,
  transform = add (lit 0) e2
}

p = eval $ transform (add (lit 5) (lit 3))
\end{verbatim}

\section{Transformations in Shallow Embeddings}
Transformations are typically regarded as the privilege of deep embeddings.
Since there is not a data structure persisting the AST, how to define transformations
becomes unclear.
It is not completely true, especially for the OO approach.
This section illustrates how to define various transformations, the missing
kind of interpretations in Gibbons and Wu's paper, in shallow embeddings.
%In fact, transformations can still be defined in shallow embedding.

\subsection{Desugaring}
Desugaring is one kind of transformations, which eliminates some
language constructs without sacrificing the expressiveness of the whole language.
In \dsl, the \emph{id} construct is such a syntactic sugar that can
be rewritten via the following formula:
$$id\ n = \overbrace{\ beside\ (fan\ 1)}^{\text{repeat }n-1\text{ times}}\ (fan\ 1)$$
which states that \emph{id n} can be represented as \emph{n} of \emph{fan 1} combined with \emph{beside}.
%Using the above formula, we can define a transformation that desugars every
%occurrence of \emph{identity} to a combination of \emph{fan} and \emph{beside}.

The following shows how to eliminate \lstinline{Id} construct in Scala:
\lstinputlisting[linerange=4-23]{../src/desugar/Circuit.scala}%APPLY:DESUGAR_SCALA
The newly introduced method \lstinline{desugar} returns a
\lstinline{Circuit} reflecting the nature of transformations. The implementation of \lstinline{desugar} for each case is
straightforward: \emph{fan n} returns itself; $beside\ c_1\ c_2$ returns a new
$beside$ with $c_1$ and $c_2$ desugared; \emph{id n} just mimics the formula.

%\weixin{define a show function in the and show the result after desugaring?}
Inspired by the Scala implementation, we can port it back to Haskell thanks to
the laziness of Haskell:
\lstinputlisting[linerange=6-21]{./code/desugar.hs}%APPLY:DESUGAR

\subsection{Sophisticated Optimizations}
Sophisticated optimizations may need to inspect multiple internal
representations of values.
Suppose we want to merge consecutive vertical wires:
$$
beside\ (id\ m)\ (id\ n) = id\ (m + n)
$$
This rewrite rule applies only when both $c_1$ and $c_2$ of $beside\ c_1\ c_2$
are $id$ and also accesses their inner value.
This kind of optimizations are hard to define in shallow embeddings since it
requires deep pattern matching on the AST.
Fortunately, Scala has built-in support for pattern matching on objects.
We can not simply decorate classes with \lstinline{case} modifier
as case class to case class extension is not supported.
Alternatively, we could manually define an \lstinline{unapply} method, a.k.a
\emph{extractor}, in the companion object:
%Here comes the implementation of the optimization:
\lstinputlisting[linerange=4-22]{../src/optimizations/Circuit.scala}%APPLY:MERGEIDS_SCALA

For other OO languages, pattern matching can still be simulated through \emph{test methods} or \emph{type test and type casts},
although they may not be as elegant as extractors~\cite{emir2007matching}.
Interested reader can refer to \ref{} for implementations using these approaches.

\begin{comment}
\paragraph{Test methods.}
We can introduce some test methods to the hierarchy for determining the class
type of object and extract information from it:
\begin{lstlisting}
  def fromId: Option[Int] = ...
\end{lstlisting}
Then
\begin{lstlisting}
(for {
  i1 <- c1.fromId; i2 <- c2.fromId
} yield new Id(i1+i2))
.orElse(Some(new Beside(t1,t2))).get
\end{lstlisting}

\paragraph{Type test and type cast.}
Each constructs is of different type. We can test its type and convert.
\begin{lstlisting}
(t1,t2) match {
  case (i1: Id,i2: Id) => new Id(i1.n + i2.n)
  case _ => new Beside(t1,t2)
}
\end{lstlisting}

\paragraph{Extractors.}
Scala supports pattern matching.
one can manually implement an extractor, a.k.a \lstinline{unapply}
method, in the companion object to tell how to destruct an object.
\begin{lstlisting}
object Id {
  def unapply(c: Identity) = Some(c.n)
}

(t1,t2) match {
  case (Id(n1),Id(n2)) => Id(n1+n2)
  case _ => Beside(t1,t2)
}
\end{lstlisting}

Note that we can not simply decorate classes with \lstinline{case} modifier to
automatically generate the \lstinline{unapply} method -
it is not possible to have a case class that extends another case class.
\end{comment}
\section{Modular EDSLs in OOP}
This section shows that procedural abstraction mechanisms provided by OO
languages improves the modularity and reusability of EDSLs, in particular
subtyping and inheritance although most OO languages blend them together in one
syntactic form.

\subsection{Subtyping}
Subtyping allows us separate interfaces and implementations, which
encapsulates details of implementations and yields flexible design of EDSLs.
Also, it forms the foundation of the \textsc{Composite} pattern,
which enables us to composite instances of subtypes where a supertype is needed,
following the Liskov substitution principle.

Figure~\ref{code:base} illustrates the use of subtyping,
where all the classes are subtype of the trait \lstinline{Circuit}.

We can easily introduce new language constructs to \dsl by adding a new class to
the hierarchy.
For example, an \lstinline{above} combinator that combines two circuits vertically:
\begin{lstlisting}
class Above(val c1: Circuit, val c2: Circuit) extends Circuit {
  def width = c1.width
}
\end{lstlisting}
The new class works seamlessly with existing classes, as it is also a subtype of
\lstinline{Circuit}. We can construct a circuit by a combination of old
and new constructs:\\
\lstinline{new Beside(new Above(new Fan(2),new Id(2)),new Id(3))}

% type refinements
% record subtyping.

\subsection{Inheritance}
Inheritance is critical for code reuse, which allows
a class to obtain functionalities from other existing classes
without duplicating the code.
Moreover, the inherited functionality can be refined through method overriding
and new functionalities can be introduced via defining new methods and
fields.

We have already shown how to reuse an EDSL through inheritance in Figure~\ref{code:desugar}.
Instead of copying and pasting \lstinline{width} definition case by case
from the old implementation to the new one, we define the new language through inheriting existing
classes, e.g. \lstinline{class Fan extends base.Fan}.
Additionally, we introduce a new method \lstinline{desugar} to the hierarchy.
and gives a default implementation in \lstinline{Circuit} and override it when needed.
Note that we do covariant type-refinements on fields of type \lstinline{Circuit} so that
we can call \lstinline{desugar} method on them.

% Overriding in FP
One may argue that similar code reuse can be simulated in FP through reusing a record instance.
In fact, it is more like composition than inheritance from the OO perspective.
Inheritance plus dynamic dispatch maximize the code reuse for composition.
Consider a simple example:

\begin{lstlisting}[language=haskell]
data T = T { f :: Int, g :: Int }
a = T { f = g a, g = 0 }
b = a { g = 1 }
\end{lstlisting}
Here, we want to reuse \lstinline{a} but ``override'' its \lstinline{g} field when defining \lstinline{b}.
However, the expected result of calling \lstinline{f b} is 1 but it is indeed 0.
The reason is that the \lstinline{f} field ``inherited'' from \lstinline{a} calls
\lstinline{g} on the old instance rather than the new one.

To make it clear, Figure~\ref{code:compare} give the corresponding OO implementation using composition
and inheritance respectively.
\begin{figure}
\begin{tabular}{lll}
\begin{minipage}{.17\textwidth}
\begin{lstlisting}
trait T {
  def f: Int
  def g: Int
}
class A extends T {
  def f = g
  def g = 0
}
\end{lstlisting}
\end{minipage}
&
\begin{minipage}{.17\textwidth}
\begin{lstlisting}
class B(a: A) extends T {
  def f = a.f
  def g = 1
}
new B(new A).f// 0
\end{lstlisting}
\end{minipage}
&
\begin{minipage}{.17\textwidth}
\begin{lstlisting}
class B extends A {
  override def g = 1
}
(new B).f// 1
\end{lstlisting}
\end{minipage}
\end{tabular}
\caption{Composition (left) vs Inheritance (right)}
\label{code:compare}
\end{figure}
In the case of inheritance, calling \lstinline{f} will execute the overridden \lstinline{g}
as \lstinline{this} points to \lstinline{B} rather than \lstinline{A}.
For compositions, a method can not be reused if when the implementation of what
they depend on changes like \lstinline{f} in \lstinline{A}, even if the logic holds.
OO languages provide these two options for reusing code.

\subsection{Extensibility of EDSLs}
As EDSLs evolve along the time, the need for new syntax and new semantics arises.
This requires host languages equipped with extensibility mechanisms.
OO languages give simple solution to the Expression Problem~\cite{eptrivially16}
and hence are suitable host languages in terms of extensibility.

We have already shown how to extend \dsl on these two dimensions modularly in an
OO language: a new language construct \lstinline{Above} and a new interpretation function \lstinline{desugar}.

Gibbons and Wu introduce new interpretations in a way similar to
Figure~\ref{code:haskell} by appending the definition to a tuple,
which requires revision on the original code.
Though they also present a modular solution based on~\ref{swierstra2008data},
the complexity of encoding increases significantly. Moreover, how to dependent
interpretation modularly is not clear.

\subsection{Discussion}
There are certain kind of methods that are not reusable using the OO approach we
presented, including transformations (or producer methods) and binary
methods as discussed in~\ref{eptrivially16}.
For transformations, we have to refine their return type in extensions and hence
the original code can not be reused, resulting code duplications.
For binary methods like equality, there is no way to refine the argument type.
It is still possible to make it reusable by using virtual types in Scala to capture the type of \lstinline{this}~\ref{zenger}.
However, this may significantly complicate the encoding.

\section{Related Work}

\paragraph{Deep and Shallow Embeddings}
Shallow embedding yields flexible and concise EDSLs while deep embedding makes
it easy to define optimizations.
There are a lot of work~\cite{svenningsson2012combining,
  Jovanovic:2014:YCD:2658761.2658771, scherr2014implicit} trying to blend these two
approaches to enjoy benefits from both by using the shallow embedding as the
surface language and generate deep embedding version to allow optimizations to
be defined.
%Hofer and Ostermann~\cite{hofer2010modular} propose to provide both embedding through implementing internal and external visitor at the same time so that clients can choose for a particular interpretation;
The OO approach we present retains the simplicity of shallow embedding while
makes it possible to directly implement optimizations.

\paragraph{Modularity of EDSLs}
Languages that solves the EP are capable of being host languages for
implementing modular EDSLs. But many of these solutions require sophisticated
type parameterization, advanced features or heavy encoding which impose
complexity to the DSL implementers or even clients and may be too complicated to use in practice.
For example, solutions based on the \textsc{Visitor}
pattern~\cite{oliveira09modular,hofer2010modular} %more references
Relatively simple solution like Object Algebras~\cite{bruno12oa} may not support
optimizations well.
Our approach, however, is simple, intuitive and supports transformations.

%\section{Overview}
\weixin{title: Evolution on DSLs?}

\begin{comment}
Weixin writes this one.

Go over jeremy's examples, maybe having only 3 diagram 
constructs instead of 5 for space reasons.

Use five constructs to show extensibility.

Think of how to introduce our tool? Using the Jeremy's examples? 
or introducing before with some other examples?

\end{comment}

%As language evolves, the need of new syntax often arises.
As DSLs evolve along the time, the demand for new syntax and and semantics may arise.
It would be good if we can introduce these new features to the DSL \emph{modularly}.
%This is actually a hard problem, known as the Expression Problem (EP)~\cite{wadler}, which
This, however, requires the host language equipped with two dimensions of
extensibility.
In this section, we argue that from extensibility perspective, OO languages are
better candidates as host languages than FP languages.
To demonstrate this, we try to extend \dsl with new syntax and new semantics in
the setting of shallow embedding. Adding new syntax is both easy for both FP
and OOP. Adding new semantics, however, is hard in FP. Although it is possible
in OOP, the solution requires some boilerplate code. We hence developed \name
for defining modular extensions easily.

\subsection{Initial System}
Before introducing any extensions, we need to rewrite the initial implementation for \dsl shown in Figure~\ref{}:

\lstinputlisting[linerange=11-26]{../src/paper/sec3/Circuit.java}%APPLY:INIT
We change the implementation language constructs from classes to interfaces and
their fields to unimplemented getter methods. The purpose of this modification
is to allow types of the fields to be refined in future, which is vital for
retaining extensibility.

\subsection{Adding New Syntax}
Shallow embedding makes it easy to add new syntax, both in FP and OOP.
Suppose that we would like to add two new constructs to the language:

\lstinputlisting[language=haskell,linerange=40-41]{./code/shallowCircuit.hs}%APPLY:SYNTAX_TYPES
\emph{stretch ns c} inserts additional wires into the circuit \emph{c} by
summing up \emph{ns} and $above c_1 c_2$ combines two circuits of the same width vertically.
With these new constructs, more complex circuits can be constructed.
For example, Figure~\ref{} shows the circuit constructed by \emph{stretch [2,2,2] (fan 3) `beside` fan 1}.

To accomplish this task in Haskell, we can simply define two more cases,
\texttt{stretch} and \texttt{above}, for the semantic function:
\lstinputlisting[language=haskell,linerange=45-46]{./code/shallowCircuit.hs}%APPLY:SYNTAX_HS
Similarly, defining two new interfaces
\texttt{Stretch} and \texttt{Above} that both extend \texttt{Circuit} and
implement the \texttt{width} method is all we need to do in Java:

\lstinputlisting[linerange=29-40]{../src/paper/sec3/Circuit.java}%APPLY:SYNTAX

\subsection{Adding New Semantics}
The new combinators, however, can not apply to arbitrary circuits - they have
some invariants in their definitions.
To make sure that a circuit is constructed correctly, we need to expand the
semantics of the language for doing such checks.

Adding new semantics, however, becomes hard for shallow embedding.
Gibbons and Wu worked around this problem in the following way:
\lstinputlisting[language=haskell,linerange=2-6]{./code/NewSemantics.hs}%APPLY:SEMANTICS_HS
which is similar to how we add \texttt{desugar} to the implementation in Section~\ref{}.
The difference is that they use a tuple instead of a record to merge semantic
domains, then define the two semantic functions simultaneously for each
case, and split the semantic functions through projections on the tuple in the end.
This solution, however, modifies existing code, breaking the requirement of EP.

Conversely, the support of covariant type-refinements and inheritance for OO
languages allows us to add new semantics in a \emph{modular} way:
%Different from records or tuples, interfaces are extensible.
\lstinputlisting[linerange=47-75]{../src/paper/sec3/Circuit.java}%APPLY:SEMANTICS

Interface \texttt{CircuitWS} extends the original interface and declare a new
semantic function \texttt{wellSized} inside.
Then all existing cases should extend both their corresponding original
implementation and \texttt{CircuitWS} and implement the new method
\texttt{wellSized}. Also, all the occurrences of \texttt{Circuit} are
refined as \texttt{CircuitWS} so that we can call \texttt{wellSized} on inner circuits returned by getters.
As Java does not support type-refinements on fields, we hence implement these
constructs as interfaces rather than classes.

\subsection{\name's Support}
There exists some boilerplate for the Java solution presented above:
\begin{itemize}
  \item Interfaces that represent language constructs should be instantiated for
    creating objects;
  \item Type-refinements should be done manually;
  \item Similar inheritance relationships have to be repeatedly stated for all interfaces.
\end{itemize}
Worse still, programmers would not get warned if they forget to extend all the
constructs with new semantics.

\name addresses all these problems through embracing family polymorphism and code instrument.
By using \name, we can refactor the extensions in the way shown in Figure\ref{}.
\lstinputlisting[linerange=80-82]{../src/paper/sec3/Circuit.java}%APPLY:FAMILY
\lstinputlisting[linerange=110-112]{../src/paper/sec3/Circuit.java}%APPLY:FAMILY_SYNTAX
\lstinputlisting[linerange=131-156]{../src/paper/sec3/Circuit.java}%APPLY:FAMILY_SEMANTICS

Note that initially all the definitions are put inside a \texttt{Family} interface.
Extensions are defined inside another interface that extends base families.
For syntax extension, we just move the definitions of new constructs into
\texttt{SyntaxExt} interface.
Semantic extension is more interesting, let us look at it in detail.
Since all the definitions are nested interfaces, their names are local and can
be reused in other families. The inheritance relationships is stated only once at family level, then \name can infer the inheritance relationships for members inside the family according to their names. Also, there is no need to
manually refine the return type, as it is done by \name. Moreover, a static \texttt{of}
method would be generated for class-like interfaces, serving as the constructor
for the interface. For example, the \texttt{Beside} inside \texttt{Semantics} after code instrumentation will look like this:

\lstinputlisting[linerange=164-176]{../src/paper/sec3/Circuit.java}%APPLY:INSTRUMENT
% TODO: not implemented yet
Family polymorphism on only gives more safety on the client code but also helps
language implementers catch bug early.
For example, if one forgets to implement, say cases from \texttt{Family}, in
\texttt{Semantics}, she would get warned they have not implemented the
\texttt{wellSized} method because class-like interfaces should not contain any
unimplemented methods except for getters.
These errors are captured by automatically re-declaring members with inheritance relationships.
%The \texttt{of} method is  returns an instance of an anonymous class
%that implements the interface with all getters implemented and its instance is returned.

% Multiple interpretation
% Dependent interpretation
% Context-sensitive interpretation

%\section{Code Generation}

%Haoyuan should write this part.
%How to generate code for family polymorphism!

In this section, we present an overview of the code that \lstinline{@Family} generates. The syntax and type system are
consistent with the Java language. We use translation functions to illustrate our code generation.
To make it clearer, we split the process of code generation into three parts, in which case we introduce two
new annotations \lstinline{@Obj} and \lstinline{@ObjOf}, which are not implemented in the source code but just help to explain
the process of translation. In our implementation, the annotation processing is a combination of the three parts, corresponding to
\lstinline{@Family}, \lstinline{@Obj} and \lstinline{@ObjOf} respectively. The translation rules are presented in Figure \ref{fig:trans1}
and Figure \ref{fig:trans2}, together with some auxiliary functions. Note that in practice, our \textsf{@Family} supports simple
generics without bounds, but for simplicity reason it is not included in the formalization.

\begin{figure}
\begin{tabular}{|l|}
\hline
\begin{lstlisting}
@Family interface New extends Base {
	interface Circuit { int width(); }
	interface Beside {
		default int width() {
			return _c1().width() + _c2().width();
		}
	}
}
\end{lstlisting} \\
\hline
$\Downarrow$ Family desugaring \\
\hline
\begin{lstlisting}
@Obj interface New extends Base {
	@Obj interface Circuit extends Base.Circuit {
		int width();
	}
	@Obj interface Beside extends
			Circuit, Base.Beside {
		Circuit _c1(); Circuit _c2();
		default int width() {
			return _c1().width() + _c2().width();
		}
	}
}
\end{lstlisting} \\
\hline
$\Downarrow$ Obj desugaring \\
\hline
\begin{lstlisting}
@ObjOf interface New extends Base {
	@ObjOf interface Circuit extends Base.Circuit {
		int width();
	}
	@ObjOf interface Beside extends
			Circuit, Base.Beside {
		Circuit _c1(); Circuit _c2();
\end{lstlisting}\vspace{-.05in}\\ \begin{lstlisting}[basicstyle=\ttfamily\small\color{red}]
		Beside withC1(Circuit val);
\end{lstlisting}\\ \begin{lstlisting}
		default int width() {
			return _c1().width() + _c2().width();
		}
	}
}
\end{lstlisting} \\
\hline
$\Downarrow$ ObjOf desugaring \\
\hline
\begin{lstlisting}
interface New extends Base {
	interface Circuit extends Base.Circuit {
		int width();
	}
	interface Beside extends Circuit, Base.Beside {
		Circuit _c1(); Circuit _c2();
		Beside withC1(Circuit val);
		default int width() {
			return _c1().width() + _c2().width();
		}
		static Beside of(Circuit _c1, Circuit _c2) {
			return new Beside() {
				Circuit c1 = _c1; Circuit c2 = _c2;
				public Circuit _c1() { return c1; }
				public Circuit _c2() { return c2; }
				public Beside withC1(Circuit val) {
					return Beside.of(val, this._c2());
				}
			};
		}
	}
	static New of() { return new New(){}; }
}
\end{lstlisting} \\
\hline
\end{tabular}
\caption{An example showing the flow of translation.}\label{fig:flow}
\end{figure}

\subsection{Flow of Translation}
Figure~\ref{fig:flow} illustrates the flow of translation using a simplified version of the Scans example, with intermediate results of desugaring shown. A base family \textsf{Base} is predefined as follows:
\begin{lstlisting}
interface Base {
	interface Circuit {}
	interface Beside extends Circuit {
		Circuit _c1(); Circuit _c2();
		Beside withC1(Circuit val);
	}
}
\end{lstlisting}
\textsf{Base} contains two members, \textsf{Circuit}, and its subtype \textsf{Beside}. \textsf{Beside} has two field methods \textsf{\_c1()} and \textsf{\_c2()} together with a wither method \textsf{withC1}. At this time, another interface \textsf{New} is defined as the new family that extends the old one, where we add an operation \textsf{width()} to the member types. This is precisely the interface that we are going to translate. In the client code, the user simply re-declares the two members with the addition of \textsf{width()}, without building subtyping relations among new and old members, and a simple annotation \textsf{@Family} will deal with all the stuff.

In the first step of translation, namely family desugaring, \textsf{@Family} collects all the members from old families, and fills in the ``extends'' to build the inheritance relations. At the same time, it re-declares all the field methods, for example, although the type name of \textsf{\_c1()} is still \textsf{Circuit}, it refers to the new member type, where we address field type refinements, and hence this new \textsf{\_c1()} has the operation \textsf{width()}. Then it delegates the work to \textsf{@Obj}. In the second phase, \textsf{@Obj} detects fluent setters and withers from the supertypes, then refines their return types to have consistency with the annotated type. In our example, the wither method \textsf{withC1} is refined to have the new \textsf{Beside} as its return type. In the last step of translation, \textsf{@ObjOf} generates constructor methods for all the interfaces, except \textsf{Circuit}, where its \textsf{width()} is an abstract method and we cannot provide a default implementation for it. In the following subsections we will respectively explain the three steps of translation in detail.

\subsection{\textsf{@Family}}
\lstinline{@Family} builds the structure of family polymorphism for the annotated type. More specifically, \lstinline{@Family} tackles two
tasks: (1) building the dependencies (subtyping relations) between new family members and old ones; (2) refining field types. To this
aim, \lstinline{@Family} re-declares all member types and field methods from the base families (see \textsf{collectMembers} and \textsf{fieldMethods} in Figure \ref{fig:trans1}). \lstinline{@Family} recognizes field methods by checking if they are non-void, no-argument methods, and if the name starts with an underscore ``\_''.  The function \textsf{collectMembers} finds all the direct member types from base families $I_1,\cdots,I_n$, creates new types with the same names for re-declaration, and builds the subtyping relations among them. The newly created types are also annotated with \lstinline{@Family}, leading to such a recursive process throughout the nested interfaces in base families. If users have already declared these member types in order to introduce new operations, behaviors or whatever, the function \textsf{combine} will integrate them to the re-declaration, by combining methods and supertypes following ``extends''. Note that we use $\textsf{ibody(}I_m$\textsf{)} to find the exact declaration body of interface $I_m$. \textcolor{red}{Haoyuan: Acutally in the implementation, \textsf{ibody} takes two arguments, one is the name of type reference (interface name $I_m$), and the other is the environment, or where the type reference $I_m$ locates.}

\subsection{\textsf{@Obj}}
The second part of annotation processing is abstracted here using the newly introduced annotation
\lstinline{@Obj}. The translation of \lstinline{@Obj} is like a preprocessing of the third part \lstinline{@ObjOf},
which generates constructor method \textsf{of} for the annotated type. Besides the generation of constructors, \lstinline{@ObjOf} also supports
a series of field and non-field methods, including getters, void and fluent setters, \textsf{with}- methods and the general \textsf{with} method.
\textcolor{red}{Haoyuan: Details on these operations? Can draw a table.} Among them, the fluent setters, \textsf{with}- methods and the general \textsf{with()} method all take the annotated type as their return types, and these types can be refined in some subtypes of the annotated type. What \lstinline{@Obj} precisely does is that it collects all these refinable methods from the super interfaces, and refines them to the annotated type, so that \lstinline{@ObjOf} will later provide a default implementation for them during the generation of constructor methods.

In Figure \ref{fig:trans2}, the first translation function presents the processing of \lstinline{@Obj}. Note that we use $\textsf{mbody(}m,I\textsf{)}$ to get the full declaration of method $m$ seen by $I$, such a method can either be directly defined in $I$ or inherited from its supertypes. Besides delegating \lstinline{@Obj} to the member types in a recursive way, it invokes the function \textsf{refine} to handle the type refinements. The definition of \textsf{refine} is shown in (1) and (2), where return types of \textsf{with-} methods and fluent setters are refined respectively. They are both state operations, and for simplicity we omit the case for general \textsf{with} method in our translation rules.

\subsection{\textsf{@ObjOf}}
The third part of translation relies on \textsf{@ObjOf}, which directly generates the static \textsf{of} method, serving as a constructor method. So it returns an instance of the annotated type. This method takes one argument for each field method from the domain of the interface. Similarly, it recursively generates \lstinline{of} methods for the annotated interface and all nested interfaces.

The translation function of \textsf{@ObjOf} is shown in the second rule of Figure \ref{fig:trans2}. It firstly uses \textsf{valid} to check whether all the abstract methods in the interface are valid; that is to say, any one of them is either a field (getter) method, a \textsf{with-} method or a setter method. We capture these patterns so that \textsf{@ObjOf} can provide default implementations for them. Then, if \textsf{valid} is satisfied, a static \textsf{of} method is generated in the interface to return an instance of it. Such an instance is implemented in the return statement using an anonymous class with auto-generated
implementations for all the methods it supports (stated above). On the other hand, users are expected to put underscores as the
prefix of field methods, and consequently \lstinline{of} identifies these field methods and takes them as its arguments.

\begin{figure*}
\begin{lstlisting}
  (*$\llbracket$*)@Family interface (*$I_m$*) extends (*$I_1,\cdots,I_n$*) {(*$\overline{meth}$*) (*$\overline{I}$*)}(*$\rrbracket$*) = (*$\llbracket$*)@Obj interface (*$I_m$*) extends (*$I_1,\cdots,I_n$*) {(*$\overline{meth'}$*) (*$\overline{I'}$*)}(*$\rrbracket$*)
\end{lstlisting}
\hspace{.3in}where $\overline{meth'}=\overline{meth}\ \cup\ $\textsf{fieldMethods(}$I_1$\textsf{)}$\ \cup\cdots\cup\ $\textsf{fieldMethods(}$I_n$\textsf{)}, $\overline{I'}$ = \textsf{newChilds(}$I_1,\cdots,I_n,\overline{I}$\textsf{)}
~\\~\\
(1) $\overline{I'}$ = \textsf{newChilds(}$I_1,\cdots,I_n,\overline{I}$\textsf{)} \textcolor{red}{(definition of newChilds)}
    \begin{itemize}
    \item $\forall I_0\in\overline{I}$, if $\exists I'_0\in\ $\textsf{collectMembers(}$I_1,\cdots,I_n$\textsf{)} and \textsf{name(}$I_0$\textsf{)} = \textsf{name(}$I'_0$\textsf{)}, then \textsf{combine(}$I_0,I'_0$\textsf{)}$\ \in\overline{I'}$
    \item Otherwise, $I_0\in\overline{I'}$
    \end{itemize}
(2) $\llbracket$\lstinline{@Family}\textsf{ interface }$I$\textsf{ extends }$\overline{I_s},\ \overline{I_t}$\textsf{ \{\}}$\rrbracket\ \in$ \textsf{collectMembers(}$I_1,\cdots,I_n$\textsf{)} \textcolor{red}{(definition of collectMembers)}
    \begin{itemize}
    \item $\forall i,$ s.t. $I\in$\textsf{ childs(}$I_i$\textsf{)},
        \begin{itemize}
        \item $\overline{I_s}$ = \textsf{suptypes(}$I_i.I$\textsf{)}
        \item $I_i.I\in\overline{I_t}$
        \end{itemize}
    \end{itemize}
(3) $\overline{I_0}$ = \textsf{childs(}$I$\textsf{)}, $\overline{I}$ = \textsf{suptypes(}$I$\textsf{)} \textcolor{red}{(definition of childs and suptypes)}
    \begin{itemize}
    \item \textsf{ibody(}$I$\textsf{)} = \textsf{interface }$I$\textsf{ extends }$\overline{I}$\textsf{ \{}$\overline{meth}\ \overline{I_0}$\textsf{\}}
    \end{itemize}
(4) $meth\in\ $\textsf{fieldMethods(}$I$\textsf{)} \textcolor{red}{(definition of fieldMethods)}
    \begin{itemize}
    \item $meth\in\ $\textsf{childs(}$I$\textsf{)}
    \item $meth$ = $I_0$\textsf{ \_m();}
    \end{itemize}
(5) \textsf{interface }$I$\textsf{ extends }$\overline{I_s}$\textsf{ \{}$\overline{meth}\ \overline{I}$\textsf{\}} = \textsf{combine(}$I_m,I_n$\textsf{)} \textcolor{red}{(definition of combine)}
    \begin{itemize}
    \item \textsf{ibody(}$I_m$\textsf{)} = \textsf{interface }$I$\textsf{ extends }$\overline{I_{s1}}$\textsf{ \{}$\overline{meth_1}\ \overline{I_1}$\textsf{\}}
    \item \textsf{ibody(}$I_n$\textsf{)} = \textsf{interface }$I$\textsf{ extends }$\overline{I_{s2}}$\textsf{ \{}$\overline{meth_2}\ \overline{I_2}$\textsf{\}}
    \item $\overline{I_s}$ = $\overline{I_{s1}}\ \cup\ \overline{I_{s2}}$
    \item $\overline{meth}$ = $\overline{meth_1}\ \cup\ \overline{meth_2}$
    \item If $\exists I_1\in\overline{I_1}, I_2\in\overline{I_2}$, \textsf{name(}$I_1$\textsf{)} = \textsf{name(}$I_2$\textsf{)}, then \textsf{combine(}$I_1,I_2$\textsf{)}$\ \in\overline{I}$
    \item Otherwise $(I\in I_1\ \Delta\ I_2)$, $I\in\overline{I}$.
    \end{itemize}
\caption{Translation of \lstinline{@Family}.}
\label{fig:trans1}
\end{figure*}

\begin{figure*}
\begin{lstlisting}
  (*$\llbracket$*)@Obj interface (*$I_0$*) extends (*$\overline{I_s}$*) {(*$\overline{meth}$*) (*$\overline{I}$*)}(*$\rrbracket$*) = (*$\llbracket$*)@ObjOf interface (*$I_0$*) extends (*$\overline{I_s}$*) {(*$\overline{meth}$*) (*$\overline{meth'}$*) (*$\overline{\llbracket\textsf{@Obj}\ I\rrbracket}$*)}(*$\rrbracket$*)
\end{lstlisting}
\hspace{.3in}where $\overline{meth'}$ = \textsf{refine(}$I_0,\overline{meth}$\textsf{)}
\begin{lstlisting}
  (*$\llbracket$*)@ObjOf interface (*$I_0$*) extends (*$\overline{I_s}$*) {(*$\overline{meth}$*) (*$\overline{I}$*)}(*$\rrbracket$*) = interface (*$I_0$*) extends (*$\overline{I_s}$*) {(*$\overline{meth}$*) ofMethod((*$I_0$*)) (*$\overline{I}$*)}
\end{lstlisting}
\hspace{.3in}with \textsf{valid(}$I_0$\textsf{)}, \textsf{of} $\notin$ \textsf{dom(}$I_0$\textsf{)}
~\\~\\
(1) $I_0$ \textsf{with}$\#m$\textsf{(}$I\ \_$\textsf{val);} $\in$ \textsf{refine(}$I_0,\overline{meth}$\textsf{)} \textcolor{red}{(part I definition of newChilds) fields with underscore, field and isField}
    \begin{itemize}
    \item \textsf{isWith(mbody(with}$\#m,I_0$\textsf{)}$,I_0$\textsf{)}
    \item \textsf{with}$\#m$ $\notin$ \textsf{dom(}$\overline{meth}$\textsf{)}
    \end{itemize}
(2) $I_0\ \_m$\textsf{(}$I\ \_$\textsf{val);} $\in$ \textsf{refine(}$I_0,\overline{meth}$\textsf{)} \textcolor{red}{(part II definition of newChilds)}
    \begin{itemize}
    \item \textsf{isSetter(mbody(}$\_m,I_0$\textsf{)}$,I_0$\textsf{)}
    \item $\_m$ $\notin$ \textsf{dom(}$\overline{meth}$\textsf{)}
    \end{itemize}
(3) \textsf{valid(}$I_0$\textsf{)} if $\forall m\ \in\ $\textsf{dom(}$I_0$\textsf{)}, let $meth$ = \textsf{mbody(}$m,I_0$\textsf{)}, one of the following cases is satisfied:  \textcolor{red}{(definition of valid)}
    \begin{itemize}
    \item \textsf{isField(}$meth$\textsf{)}, where \textsf{isField(}$I\ m$\textsf{();)} = not \textsf{special(}$m$\textsf{)}
    \item \textsf{isWith(}$meth,I_0$\textsf{)}, where \textsf{isWith(}$I'$ \textsf{with}$\#m$\textsf{(}$I\ x$\textsf{);}$,I_0$\textsf{)} = $I_0 :< I'$, \textsf{mbody(}$m,I_0$\textsf{)} = $I\ m$\textsf{();} and not \textsf{special(}$m$\textsf{)}
    \item \textsf{isSetter(}$meth,I_0$\textsf{)}, where \textsf{isSetter(}$I'$ $\_m$\textsf{(}$I\ x$\textsf{);}$,I_0$\textsf{)} = $I_0 :< I'$, \textsf{mbody(}$m,I_0$\textsf{)} = $I\ m$\textsf{();} and not \textsf{special(}$m$\textsf{)}
    \end{itemize}
(4) \textsf{ofMethod(}$I_0$\textsf{)} = \textsf{static }$I_0$\textsf{ of(}$I_1\ \_m_1,\cdots,I_n\ \_m_n$\textsf{) \{} \textcolor{red}{(definition of ofMethod)}
    \\ \textsf{return new }$I_0$\textsf{() \{}
    \\ $I_1\ m_1$ = $\_m_1$\textsf{;}$\cdots I_n\ m_n$ = $\_m_n$\textsf{;}
    \\ $I_1\ m_1$\textsf{() \{return }$m_1$\textsf{;\}}$\cdots I_n\ m_n$\textsf{() \{return }$m_n$\textsf{;\}}
    \\ \textsf{withMethod(}$I_1,m_1,I_0,\overline{e}_1$\textsf{)}$\cdots$\textsf{withMethod(}$I_n,m_n,I_0,\overline{e}_n$\textsf{)}
    \\ \textsf{setterMethod(}$I_1,m_1,I_0$\textsf{)}$\cdots$\textsf{setterMethod(}$I_n,m_n,I_0$\textsf{)}
    \\ \textsf{\};\}}
    \begin{itemize}
    \item $I_1\ m_1$\textsf{();}$\cdots I_n\ m_n$\textsf{();} = \textsf{fields(}$I_0$\textsf{)}
    \item $\overline{e}_i$ = $m_1,\cdots,m_{i-1},\_$\textsf{val}$,m_{i+1},\cdots,m_n$
    \end{itemize}
(5) $meth$ $\in$ \textsf{fields(}$I_0$\textsf{)} \textcolor{red}{(definition of fields)}
    \begin{itemize}
    \item \textsf{isField(}$meth$\textsf{)}
    \item $meth$ = \textsf{mbody(}$m^{meth},I_0$\textsf{)}
    \end{itemize}
(6) $I_0$ \textsf{with}$\#m$\textsf{(}$I\ \_$\textsf{val) \{return }$I_0$\textsf{.of(}$\overline{e}$\textsf{);\}} = \textsf{withMethod(}$I,m,I_0,\overline{e}$\textsf{)} \textcolor{red}{(definition of withMethod)}
    \begin{itemize}
    \item \textsf{mbody(with}$\#m,I_0$\textsf{)} has the form \textsf{mh;}
    \end{itemize}
(7) $I_0$ $\_m$\textsf{(}$I\ \_$\textsf{val) \{}$m$ = $\_$\textsf{val;return this;\}} = \textsf{setterMethod(}$I,m,I_0$\textsf{)} \textcolor{red}{(definition of setterMethod)}
    \begin{itemize}
    \item \textsf{mbody(}$\_m,I_0$\textsf{)} has the form \textsf{mh;}
    \end{itemize}
\caption{Translation of \lstinline{@Obj} and \lstinline{@ObjOf}.}
\label{fig:trans2}
\end{figure*}

%\section{Case Study}
%\section{Related Work}

\paragraph{Deep and Shallow Embeddings}
When designing embedded DSLs, there always exists a choice between deep
embedding and shallow embedding.
However, either approach has its own strengths and weakness.
Deep embedding uses an AST to represent programs, allowing
analyses, optimizations or further translation on the programs.
Nonetheless, the AST definition is typically not extensible, as a result
adding new language constructs becomes difficult.

Shallow embedding directly encodes the source language using the semantics of
the host language bypassing an AST, which yields flexible and concise implementations.
The dual problem in shallow embedding is that the semantic domain is fixed,
making it hard to introduce new interpretations.
Although Gibbons and Wu ~\cite{gibbons} showed that it is possible to define multiple
interpretations via tuples, the encoding requires modifications on existing code.

To enjoy the benefits of both techniques, \cite{svenningsson2012combining}
combines deep and shallow embedding together, which uses shallow embedding to
encode the surface language and translates it to a deeply embedded core language.
Similarly, Yin-Yang~\cite{Jovanovic:2014:YCD:2658761.2658771} allows user
program written in shallow embedding and automatically generates the deeply
embedded version through Scala macros. Hofer and Ostermann~\cite{hofer2010modular} propose to provide both embedding through implementing internal and external visitor at
the same time so that clients can choose for a
particular interpretation.
Our approach, however, can be viewed as a hybrid of shallow and deep embedding -
the encoding looks like shallow embedding while preserving the AST.

%Hudak 1. building DSL 2. modular DSL

% Our approach. The simplicity . The AST is preserved, allowing transformations

\paragraph{Modularity of DSLs}
How to add variants (language constructs) as well as operations
(interpretations) modularly, a.k.a the Expression Problem (EP), is a hot research
topic. There are many solutions to EP. We discuss some of them in
OOP and in FP respectively.

In OOP, the Vistor pattern is widely used for modeling languages, which is
well-known for adding new operations easily. Many works
have been done for making it possible to extend variants~\cite{oliveira07genericity,oliveira09modular}.
%In fact, the Visitor pattern has a strong connection with deep and shallow embedding.
% Specifically, the internal variant corresponds to shallow embedding whereas the external variant is equivalent to deep embedding.
Object Algebras~\cite{bruno12oa} simplifies the solution, which is an variant of
internal visitor pattern.
 % However, this approach requires heavy encoding and advanced type system features in Scala.
Another line of work is on OO decomposition. ~\cite{zenger} capture recursive
arguments using virtual types with bounds. Wang and Oliveira~\cite{eptrivially16}
uses covariant types, a common feature available in most OO languages, to achieve the same goal which is available in most OO languages.
Compared to the Visitor pattern, the OO decomposition approach is simpler and our encoding is based on ~\cite{eptrivially16}.

In FP, the two main solutions are \emph{finally
  tagless}~\cite{carette2009finally} and \emph{data types a la
  carte}~\cite{swierstra2008data} (DTC).
Finally tagless approach uses a type class to abstract over all interpretations
of the language. Concrete interpretations are given through creating a data type and
making it an instance of that type.
DTC represents language constructs separately and composes them together using
extensible sums. However, not like OO languages which come with subtyping, one
has to manually implement the subtyping machinery for variants.
Moreover, neither DTC nor finally tagless approach can add new interpretations
that depend on existing interpretations modularly like our approach.

\paragraph{Family Polymorphism}

There has been a lot of related work on family polymorphism[1], including various lightweight encodings[2,3,4,5]. In the original
paper[1], nested classes are represented as attributes of an object, which involves a dependent type system. In [4], Saito and Igarashi
proposed a lightweight variant of family polymorphism, which uses classes rather than objects to represent families. Later work in [4]
proposes a simple extension to Featherweight Generic Java[6], introducing self-type variables. There are also some existing languages,
like Scala, which supports symmetric mixin compositions and self-type annotations, hence can provide better support for encoding family
polymorphism. Our lightweight encoding relies entirely on the existing Java language without language extensions, and certainly some features from
the original paper are sacrificed, including the mismatching problem of recursive class definitions (also known as binary methods).

On the other hand, our approach uses nested interfaces to build the relationship among families and members, which is different from
path-dependent types in [1]. The work in [8] inspires us with the concept of higher-order hierarchies, and hence our \textsf{@Family} annotation
provides support for nested families. Furthermore, our lightweight encoding of family polymorphism is polished with the help of annotation processing
to generate constructor methods automatically.

[1] family polymorphism
[2] Lightweight scalable components
[3] The essence of lightweight family polymorphism
[4] Lightweight family polymorphism
[5] Lightweight dependent classes
[6] Featherweight Java
[7] On binary method
[8] Higher-order hierarchies

\paragraph{ThisType}

\paragraph{Multiple Inheritance}

%\section{Discussion and Limitations}

%Haoyuan writes this one.

%- Transformations and Binary methods (but argue that
%these types of operations are not used in shallow DSLs
%any way!).

%- Java syntax not great for DSLs.
%operator overloding; infix operators (but this is a Java limitation;
%not a limitation of our approach).

%- Limitations of the tool: separate compiltions and ...
%generics not fully implemented.

%- How to apply our approach to other OO language
%  - C\# has annotations, but can you do AST rewriting?
%  - Scala has macros;
%  - Other languages?



\paragraph{Transformations} Our approach tends to give a taste that shallow embeddings are convenient to
use in object-oriented languages, and extensible with family polymorphism. Nevertheless, shallow embeddings are also under
criticism in terms of its not supporting ASTs explicitly, leading to the fact that some operations like transformations
cannot be easily used with shallow embeddings. In Section ?(2.3) we give an example of desugaring in Java using shallow
embeddings, however when we try to apply family polymorphism for inheritance and extensibility, code duplication can be introduced.
Suppose we firstly integrate the four members \textsf{Circuit}, \textsf{Fan}, \textsf{Beside} and \textsf{Identity} in a base family,
and then define a new family that extends the base one, then inside the new member \textsf{Identity}, how can we implement the transformation \textsf{desugar()}? Certainly we cannot use \textsf{super} to call the old implementation, otherwise we are invoking old constructors of \textsf{Beside},
leading to type errors afterwards. For sure we can just copy and paste the code for the old implementation, but too much
code duplication would depress the users. Another solution is to use F-bounded polymorphism (\textcolor{red}{Haoyuan: refs. Need example?
We do have an example in the Pretty Printer case study}),
in that case code duplication is avoided, but accordingly, it arouses explosively heavy uses of types and parameterisations.

\paragraph{Binary methods} Binary methods are indeed another sort of operations that prevent users from using shallow embeddings in Java
comfortably. \textcolor{red}{Haoyuan: refs?} Java supports covariant return types, hence we can refine the types of field methods and 
simple interpretations, but with binary methods like \textsf{equals}, method subtyping in inheritance is a big problem since the argument type
is changed. In Scala, however, we can deal with binary methods with the help of ? \textcolor{red}{Haoyuan: Could you help Weixin?}.

\paragraph{Shallow or deep embeddings?} We know the fact that our approach integrates shallow embeddings and simple family polymorphism, and that in some object-oriented languages like Java,
some trouble is introduced when dealing with extensible transformations or binary methods. But on the other hand, these types of operations are not used in
shallow DSLs anyway; with a single interpretation we use shallow embeddings, but since shallow embeddings do not represent abstract syntax trees,
it is really hard to realize transformations, and similarly, with binary methods, users tend to use deep embeddings instead to build data hierarchies.

\paragraph{Limitations} Our tool has its certain limitations at a few aspects. As Lombok only provides experimental support
for separate compilation, our \textsf{@Family} does not support separate compilation at this stage, so all the related
interfaces should be included in a single Java file. On generics, our implementation of \textsf{@Family} provides support
for generic interfaces and methods without bounds, on the other hand, generic method typing is not explicitly captured
by the annotation but delegated to the Java compiler. \textcolor{red}{Haoyuan: We still need to talk about Lombok in detail?
Another limitation is that only Eclipse handler is supported.}

Regarding syntax, it would be nice to have operator overloading and infix operators in Java, so that the code could be
written in a more concise and elegant way. Nevertheless, this is limited by the programming language, and we expect our approach to be oriented to
other OO languages. \textcolor{red}{Haoyuan: How to apply our approach to other OO language? C\# has annotations, but can you do AST rewriting? Scala has macros;  Other languages?}

%\acks


%Acknowledgments, if needed.

% We recommend abbrvnat bibliography style.

\bibliographystyle{abbrvnat}
\bibliography{paper}

% The bibliography should be embedded for final submission.

\end{document}
